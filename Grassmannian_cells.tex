\documentclass{amsart}
\usepackage{amsmath,amssymb,latexsym}
\usepackage[margin=1in]{geometry}

\newtheorem{theorem}{Theorem}
\newtheorem{corollary}{Corollary}[theorem]
\newtheorem{definition}{Definition}[theorem]
\newtheorem{lemma}{Lemma}[theorem]
\newtheorem{question}{Question}[theorem]
\newtheorem{proposition}{Proposition}[theorem]

\newcommand{\bfe}{\mathbf{e}}
\newcommand{\bff}{\mathbf{f}}
\newcommand{\bfg}{\mathbf{g}}
\newcommand{\tbfe}{{\tilde\bfe}}
\newcommand{\tbff}{{\tilde\bff}}
\newcommand{\tbfg}{{\tilde\bfg}}
\newcommand{\C}{\mathbb{C}}

\newcommand{\Rep}{\mathrm{Rep}}
\newcommand{\Gr}{\mathrm{Gr}}
\renewcommand{\AA}{\mathbb{A}}
\newcommand{\ZZ}{\mathbb{Z}}
\newcommand{\Ext}{\operatorname{Ext}}
\newcommand{\Hom}{\operatorname{Hom}}

\title{Cell Decomposition of Rank 2 Quiver Grassmannians}

\author{Dylan Rupel}
\author{Thorsten Weist}

\begin{document}
\begin{abstract}
  We prove that all quiver Grassmannians for preprojective and preinjective representations of a generalized Kronecker quiver admit a cell decomposition.  
  We also provide a natural combinatorial labeling for these cells using compatible pairs in a maximal Dyck path. 
\end{abstract}
\maketitle

\section{Introduction}
-something about cluster algebras
-something about categorification and quiver Grassmannians
-something about compatible pairs and combinatorial construction of cluster variables
-statement of our results
-something about torus actions and the universal cover
-something about cell decompositions of quiver Grassmannians
-acknowledgements?


%%%%%%%%%%%%%%%%%%%%%%%%%%%%%%%%%%%%%%%%%%%%%%
\section{Torus Action on Quiver Grassmannians}
\noindent Let $Q$ be an acylic quiver with vertices $Q_0$ and arrows $Q_1$ which we denote by $\alpha:i\to j$. Moreover, let $W_{Q}$ be the free (non-abelian) group generated by $Q_1$.  We denote by $\Rep(Q)$ the category of $\C$-representations of $Q$.
\begin{definition}
The universal covering quiver $\tilde Q$ of $Q$ is given by the vertices $\tilde Q_0=Q_0\times W_{Q}$ and the arrow set $\tilde Q_1=Q_1\times W_{Q}$ where $(\alpha,w):(i,w)\to (j,w\alpha)$ for every $\alpha:i\to j$.

We say that a representation $X\in \Rep(Q)$ can be lifted (to $\tilde Q$) if there exists a representation $\tilde X\in\Rep(\tilde Q)$ such that $F_Q\tilde X=X$ where $F_Q:\Rep(\tilde Q)\to\Rep(Q)$ is the natural functor.
\end{definition}
\begin{lemma}
  Every preprojective (resp. preinjective) representation of $Q$ can be lifted to a representation of $\tilde Q$.
\end{lemma}
\begin{proof}This statement is clear for the simple representations $S_q$, $q\in Q_0$. Now every preprojective representation $X$ can be obtained when applying a sequence of BGP-reflections \cite{bgp} to a simple representation $S_{q'}$ of a quiver $Q'$ whose underlying graph is the one of $Q$. Applying BGP-reflections to a source $q$ of $Q$ corresponds to applying BGP-reflections to all vertices $(q,w)$ of $\tilde Q$. This leads the claim. The statement for preinjective representations follows in the same way.
\end{proof}
We choose a map $d:\tilde Q_0\to\ZZ$ and fix a representation $X\in\Rep(Q)$. In any case, we can consider the decomposition $X_q=\bigoplus_{w\in W_Q} X_{(q,w)}$. We define a torus action on each $X_{(q,w)}$ via $t.x_{(q,w)}=t^{d(q,w)}x_{(q,w)}$ which can be extended linearly to each $X_q$. For a fixed a subspace $U_q$, we can define the subspace $t.U_q$. In general, this torus action induces no torus action on the Quiver Grassmannians $\Gr_{\bfe}(X)$ as $t.U=(t.U_q)_{q\in Q_0}$ is no subrepresentation of $X$ for every $U\in \Gr_{\bfe}(X)$. Actually, for this the action has to satisfy $X_{\alpha}(t.U_i)\in t.U_j$ for every $\alpha:i\to j$ and every $x\in X_i$. 
\begin{lemma}Fix an integer $c_\alpha\in\ZZ$ for every $\alpha\in Q_1$.
If $X\in\Rep(Q)$ can be lifted, $d:\tilde Q_0\to\ZZ$ induces a torus action on $\Gr_\bfe(X)$ if we have $d(j,w\alpha)-d(i,w)=c_\alpha$ for all $\alpha:i\to j$ and $w\in W_{Q}$.
\end{lemma}
\begin{proof} Since $X$ can be lifted, we can write $X_\alpha:X_i\to X_j$ as block matrix consisting of linear maps $X_{(\alpha,w)}:X_{(i,w)}\to X_{(j,w\alpha)}$. Then the condition $X_{\alpha}(t.U_i)\in t.U_j$ translates into ...to be continued.
\end{proof}

\begin{lemma}
  There exists $d:\tilde Q_0\to\ZZ$ such that $d(q,w)\ne d(q',w')$ for all $q,q',w,w'$ with $\dim X_{q,w}\ne0$ and $\dim X_{q',w'}\ne0$.
\end{lemma}

\begin{theorem}
  $Gr^Q_\bfe(X)^T\cong\bigsqcup_{\tilde\bfe of type \bfe} Gr^{\tilde Q}_{\tilde\bfe}(\tilde X)$
\end{theorem}

\begin{corollary}
  affine bundles over $Gr^{\tilde Q}_{\tilde\bfe}(\tilde X)$ if $Gr^Q_\bfe(X)$ is smooth
  \[\{U\in Gr^Q_\bfe(X):\lim_{t\to0} t\cdot U\in Gr^{\tilde Q}_{\tilde\bfe}(\tilde X)\}\]
\end{corollary}

\begin{question}
  What are the ranks of these bundles?  Poincar\'e polynomials?
\end{question}


%%%%%%%%%%%%%%%%%%%%%%%%%%%%%%%%%%%%%%%%%%%%%%%%%%%%
\section{Quiver Grassmannians of $\widetilde{K(n)}$}

Define Chebyshev polynomials $u_k$ for $k\in\ZZ$ by the recursion $u_0=0$, $u_1=1$, $u_{k+1}=nu_k-u_{k-1}$.  For $m\ge1$, let $P_m$ be the preprojective representation of $K(n)$ with dimension vector $(u_m,u_{m-1})$.

Let $\tilde P_m$ be a fixed lift of $P_m$ to the universal cover $\widetilde{K(n)}$.
\begin{lemma}
  \label{le:preprojective lifts}
  There exist lifts $\tilde P_{m-1,i}$ for $1\le i\le n$ of $P_{m-1}$ to $\widetilde{K(n)}$ so that:
  \begin{enumerate}
    \item $\Hom_Q(P_{m-1},P_m)\cong\bigoplus_{i=1}^n \Hom_{\tilde Q}(\tilde P_{m-1,i},\tilde P_m)$;
    \item For any proper subset $\{i_1,\ldots,i_k\}\subset\{1,\ldots,n\}$, there exists a short exact sequence
      \[0\longrightarrow \tilde P_{m-1,i_1}\oplus\cdots\oplus\tilde P_{m-1,i_k}\longrightarrow\tilde P_m\longrightarrow \tilde P_m^{i_1,\ldots,i_k}\longrightarrow 0;\]
    \item The lifts $\tilde P_{m-1,i}$ are pairwise orthogonal. 
    \item All nontrivial proper subrepresentations of $\tilde P_m^{(k)}$ are preprojective.
  \end{enumerate}
\end{lemma}

We will always choose the subset $\{1,\ldots,k\}$ when using Lemma~\ref{le:preprojective lifts}.2 and thus we denote the cokernal simply by $\tilde P_m^{(k)}$.

\begin{lemma}
  If each $Gr^{\tilde Q}_\tbfe(\tilde P_{m-1,i})$ has a cell decomposition, then $Gr^{\tilde Q}_\tbfe(\bigoplus\tilde P_{m-1,i_j})$ has a cell decomposition.
\end{lemma}

\begin{lemma}
  \begin{enumerate}
    \item $\tilde P_m^{(n-1)}\cong\tilde P_{m-1}^{(1)}$
    \item The subrepresentation $\bigoplus_{i=1}^{k-1}\tilde P_{m-1,i}\oplus\bigoplus_{i=1}^k\tilde P_{m-2,i}\subset\bigoplus{i=1}^k\tilde P_{m-1,i}$ is in $(\tilde P_m^{(k)})^\perp$ and
      \[\Ext(\bigoplus{i=1}^k\tilde P_{m-1,i},\tilde P_m^{(k)})\cong\Ext(\tilde P_{m-1}^{(k)},\tilde P_m^{(k)})\]
      where $\tilde P_{m-1}^{(k)}$ above denotes the cokernel of the inclusion.
  \end{enumerate}
\end{lemma}

\begin{corollary}
  observe when fibers are empty
\end{corollary}

\begin{proposition}
  Consider $\psi:Gr^{\tilde Q}_\tbfe(\tilde P_m)\to\bigsqcup_{\tbff+\tbfg=\tbfe} Gr^{\tilde Q}_\tbff(\bigoplus_{i=1}^k \tilde P_{m-1,i})\times Gr^{\tilde Q}_\tbfg(\tilde P_m^{(k)}$.  Then the following hold:
  \begin{enumerate}
    \item For $V\subsetneq \tilde P_m^{(k)}$ and $U\subset\bigoplus_{i=1}^k\tilde P_{m-1,i}$, we have $\psi^{-1}(U,V)=\AA^{\langle V,\bigoplus_{i=1}^k\tilde P_{m-1,i}/U\rangle}$.
    \item If $V=\tilde P_m^{(k)}$ and the fiber is not empty, then $\psi^{-1}(U,V)$ is constant.
  \end{enumerate}
\end{proposition}
\begin{proof}
  (1) $V$ is preprojective but $\bigoplus_{i=1}^k\tilde P_{m-1,i}/U$ is not unless $U=0$

  (2) \[0\longrightarrow[V,P/U]\longrightarrow[V,U]^1\longrightarrow[V,P]^1\longrightarrow[V,P/U]^1\longrightarrow0\]
      and the middle map is surjective.
\end{proof}

\begin{theorem}
  Every quiver Grassmannian of a preprojective or preinjective representation of $K(n)$ and $\widetilde{K(n)}$ has a cell decomposition.
\end{theorem}

\begin{question}
  cells of $Gr^{\tilde Q}_\tbfe(\tilde P_m)$ are in one-to-one correspondence with certain tuples of subgraphs for smaller $\tilde P_\ell^{i_1,\ldots,i_k}$
\end{question}


%%%%%%%%%%%%%%%%%%%%%%%%%%%%%%%%%%%%%%%%%%%%%%%%%%%%%%%%%%
\section{Compatible Pairs Label Cells in $Gr^Q_\bfe(P_m)$}
\begin{thebibliography}{10}
\bibitem{bgp}
Bernstein, I.~N., Gelfand, I.~M., Ponomarev, V.~A.: Coxeter functors, and Gabriel's theorem. Russian Mathematical Surveys \textbf{28}(2), 17-32 (1973).
\end{thebibliography}

\end{document}
