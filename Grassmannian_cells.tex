\documentclass{amsart}
\usepackage{amsmath,amssymb,latexsym}
\usepackage[margin=1in]{geometry}
\usepackage{extarrows}
\input xy
\xyoption{all}

\newtheorem{theorem}{Theorem}[section]
\newtheorem{corollary}[theorem]{Corollary}
\newtheorem{definition}[theorem]{Definition}
\newtheorem{lemma}[theorem]{Lemma}
\newtheorem{question}[theorem]{Question}
\newtheorem{proposition}[theorem]{Proposition}

\newcommand{\bfe}{\mathbf{e}}
\newcommand{\bff}{\mathbf{f}}
\newcommand{\bfg}{\mathbf{g}}
\newcommand{\bfi}{\mathbf{i}}
\newcommand{\bfh}{\mathbf{h}}
\newcommand{\tbfe}{{\tilde\bfe}}
\newcommand{\tbff}{{\tilde\bff}}
\newcommand{\tbfg}{{\tilde\bfg}}
\newcommand{\tbfh}{{\tilde\bfh}}
\newcommand{\C}{\mathbb{C}}
\newcommand{\ui}{\underline i}
\newcommand{\uj}{\underline j}
\newcommand\udim{{\underline{\dim}\, }}

\newcommand{\Rep}{\mathrm{Rep}}
\newcommand{\Gr}{\mathrm{Gr}}
\renewcommand{\AA}{\mathbb{A}}
\newcommand{\kk}{\Bbbk}
\newcommand{\ZZ}{\mathbb{Z}}
\newcommand{\Ext}{\operatorname{Ext}}
\newcommand{\Hom}{\operatorname{Hom}}
\newcommand{\ses}[3]{0\rightarrow #1\rightarrow #2\rightarrow#3\rightarrow 0}
\newcommand{\Sc}[2]{\langle #1,#2\rangle}
\setcounter{MaxMatrixCols}{20}
\title{Cell Decomposition of Rank 2 Quiver Grassmannians}

\author{Dylan Rupel}
\author{Thorsten Weist}

\begin{document}
\begin{abstract}
  We prove that all quiver Grassmannians for preprojective and preinjective representations of a generalized Kronecker quiver admit a cell decomposition.  
  We also provide a natural combinatorial labeling for these cells using compatible pairs in a maximal Dyck path. 
\end{abstract}
\maketitle

\section{Introduction}
-something about cluster algebras
-something about categorification and quiver Grassmannians
-something about compatible pairs and combinatorial construction of cluster variables
-statement of our results
-something about torus actions and the universal cover
-something about cell decompositions of quiver Grassmannians
-acknowledgements?


%%%%%%%%%%%%%%%%%%%%%%%%%%%%%%%%%%%%%%%%%%%%%%
\section{Terminology and Preliminary results}
\subsection{Covering theory}
Let $Q$ be an acyclic quiver with vertices $Q_0$ and arrows $Q_1$ which we denote by $\alpha:i\to j$. We denote by $\Rep(Q)$ the category of $\C$-representations of $Q$. Moreover, let $A_Q\cong \ZZ^{Q_1}$ be the free abelian group and $W_{Q}$ be the free non-abelian group generated by $Q_1$.  
\begin{definition}
The universal abelian covering quiver $\hat Q$ of $Q$ is given by the vertices $\hat Q_0=Q_0\times A_{Q}$ and the arrow set $\hat Q_1=Q_1\times A_{Q}$ where $(\alpha,\chi):(i,\chi)\to (j,\chi+e_\alpha)$ for every $\alpha:i\to j$.

The universal covering quiver $\tilde Q$ of $Q$ is given by the vertices $\tilde Q_0=Q_0\times W_{Q}$ and the arrow set $\tilde Q_1=Q_1\times W_{Q}$ where $(\alpha,w):(i,w)\to (j,w\alpha)$ for every $\alpha:i\to j$.

We say that a representation $X\in \Rep(Q)$ can be lifted to $\hat Q$ (resp. $\tilde Q$) if there exists a representation $\hat X\in\Rep(\hat Q)$ (resp. $\tilde X\in\Rep(\tilde Q)$) such that $F_Q\hat X=X$ (resp. $G_Q \tilde X=X$) where $F_Q:\Rep(\hat Q)\to\Rep(Q)$ (resp. $G_Q:\Rep(\tilde Q)\to\Rep(Q)$) is the natural functor.
\end{definition}
Note that we also have a natural functor $H_Q:\Rep(\tilde Q)\to \Rep(\hat Q)$.
Moreover, note that every connected component of the universal covering quiver of the universal abelian covering quiver is isomorphic to a connected component of the universal covering quiver of the original quiver. The functor $F_Q$ induces a map $F_Q:\ZZ^{\hat Q_1}\to \ZZ^{Q_1}$. We say that a dimension vector $\hat \bfe$ of $\hat Q$ is compatible with $\bfe$ if $F_Q(\hat\bfe)=\bfe$. The group $A_Q$ acts on $\hat Q$ via translation inducing an action on $\Rep(\hat Q)$ and on $\ZZ^{\hat Q_1}$. We say two representation (resp. dimension vectors) are equivalent in this case. The analogous observation can also be made for $\tilde Q$.


\begin{lemma}
  Every preprojective (resp. preinjective) representation of $Q$ can be lifted to  $\hat Q$ and $\tilde Q$.
\end{lemma}
\begin{proof}This statement is clear for the simple representations $S_q$, $q\in Q_0$. Now every preprojective representation $X$ can be obtained when applying a sequence of BGP-reflections \cite{bgp} to a simple representation $S_{q'}$ of a quiver $Q'$ whose underlying graph is the same as the one of $Q$. Applying BGP-reflections to a source $q$ of $Q$ corresponds to applying BGP-reflections to all vertices $(q,\chi)$ of $\hat Q$ (resp. $(q,w)$ of $\tilde Q$). This leads the claim. The statement for preinjective representations follows in the same way.
\end{proof}

\subsection{Bialynicki-Birula decomposition}
The aim of this section is to define a torus action on quiver Grassmannians which can be used to simplify the calculation of homological invariants in general. If the quiver Grassmannian is smooth, which is for instance the case for exceptional representations by \cite{cr}, it can also be used to stratify the quiver Grassmannians using the results of Bialynicki-Birula. More detailed, let $X$ be a smooth projective variety with a $\C^\ast$-action. For a connecting component of the fixed point set $C\subset X^{\C^\ast}$, we define its attracting set as
\[\mathrm{Att}(C):=\{y\in X\mid \lim_{t\to 0}t.y\in C\}\]

Then we have the following statement, \cite[Section 4]{bb}:
\begin{theorem}Let $\coprod_{i=1}^r C_i= X^{\C^\ast}$ be the decomposition into connected components. Then $\mathrm{Att}(C_i)$ is a locally closed smooth $\C^\ast$-invariant subvariety of $X$ where $C_i$ is a subvariety of $\mathrm{Att}(C_i)$. Moreover, we have $X=\coprod_{i=1}^r\mathrm{Att}(C_i)$ and
the natural map $\gamma_i:\mathrm{Att}(C_i)\to C_i$ is an affine vector bundle. 

In particular, we have $\chi(X)=\chi(X^{\C^\ast})$ for the Euler characteristic.
\end{theorem}


\section{Torus Actions on Quiver Grassmannians}
\noindent 

Fix a vector space $X$ of dimension $n$ and let $d\leq n$. We first consider a natural torus action on usual Grassmannians coming along with any direct sum decomposition of the vector space $X$. Then we generalize the concept to quiver Grassmannians, but we will see that the torus fixed point set can be calculated in an analogous manner. Thus assume that $X=\bigoplus_{j=1}^l X_j$ is a direct sum decomposition of $X$. We choose a basis $\mathcal B=\{v_1,\ldots,v_n\}$ of $X$ which is compatible with this decomposition, i.e. there exist number $1\leq j_1\leq j_2\ldots\leq j_{l-1}\leq n$ such that
\[v_1,\ldots,v_{j_1}\in X_1,\,v_{j_1+1},\ldots,v_{j_2}\in X_2,\ldots,v_{j_{l-1}+1},\ldots,v_{n}\in X_l\]
Moreover, we choose a map $d:\{1,\ldots,n\}\to\mathbb{Z}$ such that $d(i)=d(i')$ if $v_i,v_{i'}\in X_{j}$ for some $j$. Then there exists a $\C^\ast$-action on $X$ when extending the definition $t.v_i:=t^{d(j)}v_i$ if $v_i\in X_j$ linearly to $X$. This naturally induces an action of $\C^\ast$ on the usual Grassmannians $\Gr_d(X)$. We can write every $U\in\Gr_d(X)$ uniquely in row-echelon form with respect to the basis $\mathcal B$, i.e. there exists a unique sequence $1\leq i_1\leq\ldots\leq i_d\leq n$ and a unique matrix representation of $U$ of the form
\[M(U):=\begin{pmatrix}\ast&\ldots &\ast &1&0 &\ldots&\ldots&\ldots&\ldots&\ldots&\ldots&\ldots&\ldots&\ldots&0\\\ast&\ldots &\ast&0&\ast&\ldots &\ast&1&0&\ldots&\ldots&\ldots&\ldots&\ldots&0\\\vdots &\ddots&\vdots&\vdots&\vdots &\ddots&\vdots&0&\ast&\ddots&\ddots&\ddots&\ddots&\ddots&\vdots\\\vdots &\ddots&\vdots&\vdots&\vdots &\ddots&\vdots&\vdots&\ddots&\ddots&\ddots&\ddots&\ddots&\ddots&\vdots\\\ast&\ldots &\ast&0&\ast&\ldots &\ast&0&\ast&\ldots&\ast&1&0&\ldots&0\end{pmatrix}\in M_{d,n}(k)\]
where the unit vectors are in the columns $i_1,\ldots,i_d$. All matrix representations of this form define a Schubert cell $X_{\bfi}$ where $\bfi=(i_1,\ldots,i_d)$. The $\C^\ast$-action is then given by multiplying the $w^{\mathrm{th}}$ column with $t^{d(w)}$ where $v_w\in X_j$, i.e. we have
\[(t.M(U))_{u,w}=t^{d(w)}\cdot M(U)_{u,w}.\]
Every Schubert cell is invariant under the torus action, i.e. $t.M(U)\in X_\bfi$ for all $M(U)\in X_{\bfi}$. 
\begin{lemma}\label{usualGrass}
Choose $d$ in such a way that $d(i)\neq d(i')$ if $v_i\in X_j$, $v_{i'}\in X_{j'}$ and $j\neq j'$.
Then we have $U\in X_{\bfi}^{\C^\ast}$ if and only if $U=\bigoplus_{i=1}^l U\cap X_i$.
\end{lemma}
\begin{proof}If $U=\bigoplus_{i=1}^l U\cap X_i$, we have 
\[t.U=\bigoplus_{i=1}^l t.(U\cap X_i)=\bigoplus_{i=1}^l t^{d(j_i+1)}(U\cap X_i)=U\]
where $j_0=0$.

Assume that $U$ is a torus fixed with $M(U)\in X_{\bfi}$. Fix the $w^{\mathrm{th}}$ column $M(U)_w$ and assume that $v_{i_w}\in X_s$. As the matrix representation $M(U)$ is uniquely determined, by the assumption on $d$, it follows that all matrix coefficients corresponding to a basis element $v_i$ such that $v_i\notin X_s$ are zero. This means that $\widehat{M(U)}_w\in X_s$ where $\widehat{M(U)}_w$ is the vector of $U$ corresponding to $M(U)_w$. As the colums are linear independent, this already shows the claim.
\end{proof}
The next step is to generalize this to quiver Grassmannians.

Let $Q$ be an acyclic quiver. We choose a map $d:\hat Q_0\to\ZZ$ and fix a representation $X\in\Rep(Q)$ which can be lifted to $\hat Q$. We can consider the decomposition $X_q=\bigoplus_{(q,\chi)\in A_Q} X_{(q,\chi)}$. We define a torus action on each $X_{(q,\chi)}$ via $t.x_{(q,\chi)}=t^{d(q,\chi)}x_{(q,\chi)}$ which can be extended linearly to each $X_q$. For a fixed a subspace $U_q$, we can define the subspace $t.U_q$. In general, this torus action induces no torus action on the Quiver Grassmannians $\Gr_{\bfe}(X)$ as $t.U=(t.U_q)_{q\in Q_0}$ is no subrepresentation of $X$ for every $U\in \Gr_{\bfe}(X)$. Actually, such an action has to satisfy $X_{\alpha}(t.U_i)\in t.U_j$ for every $\alpha:i\to j$. 
\begin{lemma}Fix an integer $c_\alpha\in\ZZ$ for every $\alpha\in Q_1$.
If $X\in\Rep(Q)$ can be lifted, $d:\hat Q_0\to\ZZ$ induces a torus action on $\Gr_\bfe(X)$ if we have $d(j,\chi+e_\alpha)-d(i,\chi)=c_\alpha$ for all $\alpha:i\to j$ and $\chi\in A_{Q}$.
\end{lemma}
\begin{proof} Let $U\in \Gr_\bfe(X)$. Since $X$ can be lifted, we can write $X_\alpha:X_i\to X_j$ as block matrix consisting of linear maps $X_{(\alpha,\chi)}:X_{(i,\chi)}\to X_{(j,\chi+e_\alpha)}$. Let $u_{(i,\chi)}\in U_{(i,\chi)}$ and $\alpha:i\to j\in Q_1$. Then we have $X_\alpha(u_{(i,\chi)})\in U_{(j,\chi+e_\alpha)}$ because $X$ can be lifted to the universal cover. Then it follows
\[X_\alpha(t.u_{(i,\chi)})=X_\alpha(t^{d(i,\chi)}u_{(i,\chi)})=t^{d(i,\chi)}X_\alpha(u_{(i,\chi)})=t^{d(i,\chi)}u_{(j,\chi+e_\alpha)}=t^{-c_\alpha}t.u_{(j,\chi+e_\alpha)}\]
for some $u_{(j,\chi+e_\alpha)}\in U_{(j,\chi+e_\alpha)}$.
Now we have \[\bigoplus_{\chi\in A_Q} t.U_{(j,\chi)}=\bigoplus_{\chi\in A_Q}t^{c(j,\chi)}t.U_{(j,\chi)}\]
for every $t\in\C^\ast$ if $c(j,\chi)$ is constant along $(j,\chi)\in \{j\}\times A_Q$. This yields  
\[t.U_j=\bigoplus_{\chi\in A_Q} t.U_{(j,\chi)}=\bigoplus_{\chi\in A_Q}t^{-c_\alpha}t.U_{(j,\chi)}\]
for every $t\in \C^\ast$.
\end{proof}


\begin{lemma}\label{degreecondition} Let $X$ be an indecomposable representation of $\hat Q$. There exists $d:\mathrm{supp} (X)\to\ZZ$ and $c_\alpha\in\mathbb N_+$ for each $\alpha\in Q_1$ such that $d(i,\chi)\ne d(j,\chi')$ if $i\neq j$ or $\chi\neq \chi'$ and such that $d(j,\chi')-d(i,\chi)=c_\alpha$ if and only if $\chi'=\chi+e_\alpha$ and $(i,\chi),\,(j,\chi+e_\alpha)\in \mathrm{supp}(X)$. 
\end{lemma}
\begin{proof}
Since $X$ is finite-dimensional, $\mathrm{supp}( X)$ is a connected finite tree. In order to prove the statement, we may assume that $d(q,0)=0$ and that $(q,0)\in \mathrm{supp}(X)$. We now fix numbers $c_\alpha$ and show that the two conditions are satisfied. To do so, let $K$ be the maximal length of a path in $\mathrm{supp}(X)$ and assume that $|Q_1|=n$. Set $c_{\alpha_1}=1$ and inductively 
\[c_{\alpha_n}> (K+1)\sum_{i=1}^{n-1}c_{\alpha_i}.\]
Note that if $A_Q$ is the free abelian group in the generators $\alpha_1,\ldots,\alpha_n$ and $f:A_Q\to \ZZ$ the group homomorphism defined by $f(\alpha)=c_\alpha$ for all $\alpha\in Q_1$,  we have $d(q,\chi)=f(\chi)$. In particular, there exist $d_i$ with $|d_i|\leq K$ such that we have
\[d(q,\chi)=\sum_{i=1}^nd_ic_{\alpha_i}.\]

First note that we obviously have $d(j,\chi+\alpha)-d(i,\chi)=c_\alpha$. Now assume that $d(i,\chi)=d(j,\chi')$ which means that $d(i,\chi)=\sum_{i=1}^{n}d_ic_{\alpha_i}=\sum_{i=1}^{n}e_ic_{\alpha_i}=d(j,\chi')$
which implies
\[\sum_{i=1}^{n-1}(d_i-e_i)c_{\alpha_i}=(e_{n}-d_{n})c_{\alpha_{n}}.\]
Since we have $|d_i-e_i|\leq K$, we obtain $$|\sum_{i=1}^{n-1}(d_i-e_i)c_{\alpha_i}|\leq K\sum_{i=1}^{n-1}c_{\alpha_i}<c_{\alpha_n}$$
This inductively yields $d_i=e_i$ for $i=n,\ldots,1$ by the choice of $c_{\alpha_{i}}$. An analogous proof applies to show that $d(j,\chi')-d(i,\chi)\neq c_\alpha$ if $\chi'\neq \chi+e_\alpha$ using that, actually, $c_{\alpha_n}> (K+1)\sum_{i=1}^{n-1}c_{\alpha_i}$. 

\end{proof}
In the following, we say that $d:\mathrm{supp}(X)\to\ZZ$ satisfies the degree condition for $X$ if it has the properties of Lemma \ref{degreecondition}.



\begin{theorem}Let $X$ be a representation which can be lifted to $\hat Q$ and choose $d:\mathrm{supp}(X)\to \ZZ$ such that it satisfies the degree condition for $X$. Then the torus action of $T$ on $\bigoplus_{q\in Q_0}X_q$ induced by $t.x_{(q,\chi)}=t^{d(q,\chi)}x_{(q,\chi)}$ for $x_{(q,\chi)}\in X_{(q,\chi)}$ induces
 a torus action on $Gr_\bfe^Q(X)$ such that
  $$Gr^Q_\bfe(X)^T\cong\bigsqcup_{[\hat\bfe]} Gr^{\hat Q}_{\hat\bfe}(\hat X)$$ where $\hat\bfe$ runs through all equivalence classes of dimension vectors compatible with $\bfe$.
\end{theorem}
\begin{proof}
A representation $U\in\Gr_{\bfe}(X)$ is a torus fixed point if and only if $t.U=U $ for all $t\in\C^\ast$, i.e. $t.U_q=U_q$ for all $q\in Q_0$, $t\in\C^\ast$. Thus, apart from being a subrepresentation of $X$, it is a fixed point of the torus actions on the usual Grassmannians $\Gr_{\bfe_q}(X_q)$ induced by the decompositions $X_q=\bigoplus_{\chi\in A_Q} X_{q,\chi}$. A subrepresentation $U\subset X$ can be lifted to the universal abelian covering if and only if
\[X_q\cap U_q=\bigoplus_{\chi\in A_Q} X_{(q,\chi)}\cap U_{(q,\chi)}=\bigoplus_{\chi\in A_Q}U_{(q,\chi)}.\]
Thus we can apply Lemma \ref{usualGrass} to obtain the claim.
\end{proof}
The next step is to iterate the torus action keeping in mind the following idea: every representation, which lifts to the universal covering quiver, also lifts to the universal abelian covering quiver and also to the iterated universal abelian covering quivers, i.e. to $\hat Q^{(k)}:=\widehat{\hat Q^{(k-1)}}$ with $\hat Q=\hat Q^{(1)}$. Now it is straightforward to check that there exist natural surjective morphisms $f_k:\tilde Q\to \hat Q^{(k)}$ which become injective on finite subquivers if $k\gg 0$, see also \cite[Section 3.4]{wei}. Since the support of $X$ is finite as a representation of $\tilde Q$, we can find $k\geq 1$ such that the full subquiver with vertices $\mathrm{supp}(X)\subseteq \hat Q^{(k)}_0$ is a tree. Thus, if $\hat X^{(k)}$ is the lift of $X$ to $\hat Q^{(k)}$, there exists a torus action on the vector spaces $\hat X^{(k)}_q$ which induce a torus action on the quiver Grassmannian $Gr_{\bfe}^{\hat Q^{(k)}}(\hat X^{(k)})$. If we denote the iterated torus fixed points by $Gr^Q_\bfe(X)^{T^{(k)}}$. We obtain:
\begin{corollary}
Let $X$ be a representation which can be lifted to $\tilde Q$. Then there exists an iterated torus action such that
  $$Gr^Q_\bfe(X)^{T^{(k)}}\cong \bigsqcup_{[\hat\bfe^{(k-1)}]} Gr_{\hat\bfe^{(k-1)}}^{\hat Q^{(k-1)}}(\hat X^{(k-1)})^T\cong \bigsqcup_{[\hat\bfe^{(k)}]} Gr^{\hat Q}_{\hat\bfe^{(k)}}(\hat X^{(k)})\cong \bigsqcup_{[\tilde\bfe]} Gr^{\tilde Q}_{\tilde\bfe}(\tilde X)$$ where $\hat\bfe^{(k-1)},\,\hat\bfe^{(k)},\,\tilde\bfe$ run through all equivalence classes of dimension vectors compatible with $\bfe$.
\end{corollary}
%\begin{corollary}
%  affine bundles over $Gr^{\tilde Q}_{\tilde\bfe}(\tilde X)$ if $Gr^Q_\bfe(X)$ is smooth
%  \[\{U\in Gr^Q_\bfe(X):\lim_{t\to0} t\cdot U\in Gr^{\tilde Q}_{\tilde\bfe}(\tilde X)\}\]
%\end{corollary}

\begin{question}
  What are the ranks of the bundles over the fixed point set?  Poincar\'e polynomials?
\end{question}


%%%%%%%%%%%%%%%%%%%%%%%%%%%%%%%%%%%%%%%%%
\section{Representation Theory of $K(n)$}

\noindent Denote by $K(n)$ the \emph{$n$-Kronecker quiver} $1\stackrel{n}{\longleftarrow}2$ with vertices $K(n)_0=\{1,2\}$ and $n$ arrows from vertex $2$ to vertex $1$. By $S_1$ and $S_2$ we denote the BGP-reflection functor corresponding to the sink and source of $K(n)$ respectively \cite{bgp}. Note that the quiver $K(n)$ itself does not change when applying the reflection functors, but the vertices $1$ and $2$ are exchanged. Let $$\mathcal S:=\begin{cases}S_1\text{ if }1 \text{ is a source }\\S_2\text{ if }$2$ \text{ is a source }\end{cases}$$

Define Chebyshev polynomials $u_k$ for $k\in\ZZ$ by the recursion $u_0=0$, $u_1=1$, $u_{k+1}=nu_k-u_{k-1}$.
\begin{theorem}
  For each $m\ge1$, there exist unique (up to isomorphism) indecomposable rigid representations $P_m=\mathcal S^{m-1} P_1$ and $I_m=\mathcal S^{-m+1} I_1$ of $K(n)$ with dimension vectors $(u_m,u_{m-1})$ and $(u_{m-1},u_m)$ respectively. Moreover, any rigid representation of $K(n)$ is isomorphic to one of the form $P_m^{a_1}\oplus P_{m+1}^{a_2}$ or $I_m^{a_1}\oplus I_{m+1}^{a_2}$ for some $m\ge1$ and some $a_1,a_2\ge0$.
\end{theorem}
The representations $P_m$ are called the \emph{preprojective} representations of $K(n)$ and the representations $I_m$ are called \emph{preinjective}.

\begin{lemma}
  For any $m\ge1$, we have 
  \[\dim\Hom(P_m,P_m)=1,\quad \dim\Hom(P_m,P_{m+1})=n,\quad \dim\Hom(P_{m+1},P_m)=0.\]  
  Moreover, for $1\le k<n$ and linearly independent functions $f_1,\ldots,f_k\in\Hom(P_m,P_{m+1})$, the map $[f_1\ \cdots\ f_k]:P_m^k\to P_{m+1}$ is injective. 
\end{lemma}
\begin{proof} 
  Note that we have an Auslander-Reiten sequence
  \begin{equation}
  \label{eq:AR sequence}
    0\longrightarrow P_{m-1}\longrightarrow P_m^n\stackrel{[f_1\ \cdots\ f_n]}{\longrightarrow} P_{m+1}\longrightarrow 0
  \end{equation}
  with irreducible homomorphisms $f_i$ which are also linearly independent. 
  Thus they do not factor through a representation $Z\neq P_{m}$ which already means that they are injective, see for instance \cite[Lemma 1.6]{ass}. 
  If we pick $k<n$ linear independent homomorphisms $f_{i_1},\ldots, f_{i_k}$, this also means that $[f_{i_1}\ \cdots\ f_{i_k}]:P_m^k\to P_{m+1}$ is injective. 
  Indeed otherwise, at least one of the homomorphisms were forced to factor through a representation $Z\neq P_{m}$.
\end{proof}

Write $P_{m+1}^{\{f_1,\ldots,f_k\}}$ for the cokernel of the map $[f_1\ \cdots\ f_k]:P_m^k\to P_{m+1}$, i.e. we have a short exact sequence
\begin{equation}
\label{eq:truncated preprojectives}
  0\longrightarrow P_m^k\stackrel{[f_1\ \cdots\ f_k]}{\longrightarrow} P_{m+1}\longrightarrow P_{m+1}^{\{f_1,\ldots,f_k\}}\longrightarrow 0.
\end{equation}

\begin{lemma}
  For $m\ge1$ and linearly independent functions $f_1,\ldots,f_k\in\Hom(P_m,P_{m+1})$, the following hold:
  \begin{enumerate}
    \item $\Hom(P_m,P_{m+1}^{\{f_1,\ldots,f_k\}})$ is naturally isomorphic to $\Hom(P_m,P_{m+1})/\kk f_1\oplus\cdots\oplus\kk f_k$;
    \item $\Ext(P_{m+1}^{\{f_1,\ldots,f_k\}},P_m)$ is naturally isomorphic to $\kk f_1\oplus\cdots\oplus\kk f_k$;
    \item $\Hom(P_{m+1}^{\{f_1,\ldots,f_k\}},P_m)=0$.
  \end{enumerate}
\end{lemma}
\begin{proof}
  Applying the functor $\Hom(P_m,-)$ to the sequence \eqref{eq:truncated preprojectives} gives the exact sequence
  \[0\longrightarrow \Hom(P_m,P_m^k)\stackrel{[f_1\ \cdots\ f_k]\circ}{\longrightarrow} \Hom(P_m,P_{m+1})\longrightarrow \Hom(P_m,P_{m+1}^{\{f_1,\ldots,f_k\}})\longrightarrow 0,\]
  establishing (1).  
  
  To see (2), we apply the functor $\Hom(-,P_m)$ to the sequence \eqref{eq:truncated preprojectives} to get an isomorphism $\Hom(P_m^k,P_m)\cong\Ext(P_{m+1}^{\{f_1,\ldots,f_k\}},P_m)$.
  Observe that $\Hom(P_m^k,P_m)$ is spanned by the coordinate projections $\pi_i:P_m^k\to P_m$ for $1\le i\le k$ and each of these gives rise to a commutative diagram
  \[\xymatrix{ & 0\ar[d] & 0\ar[d] & & \\ & P_m^{k-1}\ar[d]\ar@{=}[r]& P_m^{k-1}\ar[d]^{[f_1\ \cdots\ \widehat{f_i}\ \cdots\ f_k]}& & \\
    0\ar[r]& P_m^k\ar[r]^{[f_1\ \cdots\ f_k]}\ar[d]^{\pi_i}\ar@{}[dr]|(.35){\ulcorner}& P_{m+1}\ar[r]\ar[d]& P_{m+1}^{\{f_1,\ldots,f_k\}}\ar[r]\ar@{=}[d]& 0 \\
    0\ar[r]& P_m\ar[r]^(.35){f_i}\ar[d]& P_{m+1}^{\{f_1,\ldots,\widehat{f_i},\ldots,f_k\}}\ar[r]\ar[d]& P_{m+1}^{\{f_1,\ldots,f_k\}}\ar[r]& 0\\
     & 0 & 0 & & }\]
  The bottom row of this diagram is the extension in $\Ext(P_{m+1}^{\{f_1,\ldots,f_k\}},P_m)$ naturally identified with the generator $f_i$ of $\kk f_1\oplus\cdots\oplus\kk f_k$.

  Part (3) follows immediately from (2) and the observation that 
  \[\langle P_{m+1}^{\{f_1,\ldots,f_k\}},P_m\rangle=\langle P_{m+1},P_m\rangle-k\langle P_m,P_m\rangle=-k.\]
\end{proof}

\begin{lemma}
  Let $f_1,\ldots,f_n\in\Hom(P_m,P_{m+1})$ be linearly independent morphisms and write $g:P_{m-1}\to P_m$ for the composition of the inclusion from \eqref{eq:AR sequence} with the projection $\pi_n:P_m^n\to P_m$ to the $n^{th}$ factor.  Then $P_{m+1}^{\{f_1,\ldots,f_{n-1}\}}\cong P_m^{\{g\}}$.
\end{lemma}
\begin{proof}
  We have the following commutative diagram
  \[\xymatrix{0\ar[r]& P_m^{n-1}\ar@{=}[d]\ar[r]& P_m^n\ar[r]^{\pi_n}\ar[d]^{[f_1\ \cdots\ f_n]}\ar@{}[dr]|(.65){\lrcorner}& P_m\ar[r]\ar[d]^{f_n}& 0 \\
  0\ar[r]& P_m^{n-1}\ar[r]^{[f_1\ \cdots\ f_{n-1}]}& P_{m+1}\ar[r]& P_{m+1}^{\{f_1,\ldots,f_{n-1}\}}\ar[r]& 0}\]
  Since the vertical morphism on the left is an equality, the right hand square is a pullback.  
  But this implies the kernels of the vertical morphisms coincide.  
  The middle vertical morphism fits into the exact sequence \eqref{eq:AR sequence} and so the kernel is $P_{m-1}$.
  It follows that there is a morphism $g:P_{m-1}\to P_m$ so that $P_{m+1}^{\{f_1,\ldots,f_{n-1}\}}\cong P_m^{\{g\}}$.
\end{proof}

For any nonempty subset $I\subset[1,k]$, say with $|I|=\ell$, we get an exact sequence
\[0\longrightarrow P_m^\ell\stackrel{[f_i]_{i\in I}}{\longrightarrow} P_{m+1}^{\{f_j\}_{j\in[1,k]\setminus I}}\longrightarrow P_{m+1}^{\{f_1,\ldots,f_k\}}\longrightarrow 0.\]
Each such sequence has the following almost-split property for subrepresentations of $P_{m+1}^{\{f_1,\ldots,f_k\}}$.
\begin{lemma}
  For $m\ge1$ and linearly independent functions $f_1,\ldots,f_k\in\Hom(P_m,P_{m+1})$, given any proper subrepresentation $V\subsetneq P_{m+1}^{\{f_1,\ldots,f_k\}}$, the upper pullback sequence
  \[\xymatrix{0\ar[r] & P_m^\ell\ar@{=}[d]\ar[r] & V^I\ar[r]\ar[d]\ar@{}[dr]|{\lrcorner} & V\ar[d]\ar[r] & 0 \\
  0\ar[r]& P_m^\ell\ar[r]^(.35){[f_i]_{i\in I}} & P_{m+1}^{\{f_j\}_{j\in[1,k]\setminus I}}\ar[r]& P_{m+1}^{\{f_1,\ldots,f_k\}}\ar[r]& 0}\]
  splits for every $I\subset[1,k]$.
\end{lemma}
\begin{proof}
  When $I=\emptyset$, we have $\ell=0$ and there is nothing to show, so assume $I$ is a nontrivial subset of $[1,k]$.  
  
  We proceed by simultaneous induction on $k$ and $\ell=|I|$.  For any nonzero $f\in\Hom(P_m,P_{m+1})$ the exact sequence
  \[\xymatrix{0\ar[r] & P_m\ar[r]^f & P_{m+1}\ar[r] & P_{m+1}^{\{f\}}\ar[r] & 0}\]
  is almost split, giving the claim in the case $k=\ell=1$.  For $k>1$ and a nontrivial proper subset $I\subsetneq[1,k]$, we have the following commutative diagram
  \[\xymatrix{ & & 0\ar[d] & 0\ar[d] & \\   & 0\ar[r]\ar[d] & P_m^{k-\ell}\ar@{=}[r]\ar[d]^{[f_j]_{j\in[1,k]\setminus I}} & P_m^{k-\ell}\ar[d]^{[f_j]_{j\in[1,k]\setminus I}}\ar[r] & 0 \\
  0\ar[r] & P_m^\ell\ar@{=}[d]\ar[r]^{[f_i]_{i\in I}} & P_{m+1}\ar[r]\ar[d]\ar@{}[dr]|{\lrcorner} & P_{m+1}^{\{f_i\}_{i\in I}}\ar[d]\ar[r] & 0 \\
  0\ar[r]& P_m^\ell\ar[r]^(.35){[f_i]_{i\in I}}\ar[d] & P_{m+1}^{\{f_j\}_{j\in[1,k]\setminus I}}\ar[r]\ar[d] & P_{m+1}^{\{f_1,\ldots,f_k\}}\ar[r]\ar[d]& 0\\ & 0 & 0 & 0 & }\]
  A proper subrepresentation $V\subsetneq P_{m+1}^{\{f_1,\ldots,f_k\}}$ gives rise, via pullbacks, to the following commutative diagram
  \[\xymatrix{ & & 0\ar[d] & 0\ar[d] & \\   & 0\ar[r]\ar[d] & P_m^{k-\ell}\ar@{=}[r]\ar[d]^{[f_j]_{j\in[1,k]\setminus I}} & P_m^{k-\ell}\ar[d]^{[f_j]_{j\in[1,k]\setminus I}}\ar[r] & 0 \\
  0\ar[r] & P_m^\ell\ar@{=}[d]\ar[r]^{[f_i]_{i\in I}} & V^{[1,k]}\ar[r]\ar[d]\ar@{}[dr]|{\lrcorner} & V^{[1,k]\setminus I}\ar[d]\ar[r] & 0 \\
  0\ar[r]& P_m^\ell\ar[r]^(.35){[f_i]_{i\in I}}\ar[d] & V^I\ar[r]\ar[d] & V\ar[r]\ar[d]& 0\\ & 0 & 0 & 0 & }\]
  in which the middle vertical sequence and the middle horizontal sequence split by induction.  
  But the images of $[f_i]_{i\in I}$ and $[f_j]_{j\in[1,k]\setminus I}$ intersect trivially inside $V^{[1,k]}$ and thus the lower horizontal sequence must split as well, establishing the claim for $I\subsetneq[1,k]$.

  For $I=[1,k]$, we consider the commutative diagram
  \[\xymatrix{ & & 0\ar[d] & 0\ar[d] & \\   & 0\ar[r]\ar[d] & P_{m+1}\ar@{=}[r]\ar[d] & P_{m+1}\ar[d]\ar[r] & 0 \\
  0\ar[r] & P_m^{k-1}\oplus P_m\ar@{=}[d]\ar[r] & P_{m+1}\oplus P_{m+1}\ar[r]\ar[d]\ar@{}[dr]|{\lrcorner} & P_{m+1}^{\{f_1\ \cdots\ f_{k-1}\}}\oplus P_{m+1}^{\{f_k\}}\ar[d]\ar[r] & 0 \\
  0\ar[r]& P_m^k\ar[r]^{[f_1\ \cdots\ f_k]}\ar[d] & P_{m+1}\ar[r]\ar[d] & P_{m+1}^{\{f_1,\ldots,f_k\}}\ar[r]\ar[d]& 0\\ & 0 & 0 & 0 & }\]
  A proper subrepresentation $V\subsetneq P_{m+1}^{\{f_1,\ldots,f_k\}}$ then gives rise to the following commutative diagram
  \[\xymatrix{ & & 0\ar[d] & 0\ar[d] & \\   & 0\ar[r]\ar[d] & V^{[1,k]}\ar@{=}[r]\ar[d]^{\Delta} & V^{[1,k]}\ar[d]\ar[r] & 0 \\
  0\ar[r] & P_m^{k-1}\oplus P_m\ar@{=}[d]\ar[r] & V^{[1,k]}\oplus V^{[1,k]}\ar[r]\ar[d]\ar@{}[dr]|{\lrcorner} & V^{[1,k]\setminus I}\oplus V^I\ar[d]\ar[r] & 0 \\
  0\ar[r]& P_m^k\ar[r]\ar[d] & V^{[1,k]}\ar[r]\ar[d] & V\ar[r]\ar[d]& 0\\ & 0 & 0 & 0 & }\]
  in which the middle vertical sequence splits and the middle horizontal sequence is split by induction.  
  But the image of $[f_1\ \cdots\ f_k]:P_m^{k-1}\oplus P_m\to V^{[1,k]}\oplus V^{[1,k]}$ and the diagonal $\Delta:V^{[1,k]}\to V^{[1,k]}\oplus V^{[1,k]}$ intersect trivially because $f_1,\ldots,f_k$ are linearly independent.
  As above the lower horizontal sequence also splits, establishing the claim for $I=[1,k]$.
\end{proof}


%%%%%%%%%%%%%%%%%%%%%%%%%%%%%%%%%%%%%%%%%%%%%%%%%%%%
\section{Quiver Grassmannians of $\widetilde{K(n)}$}
\noindent 
Let $\tilde P_m$ be a fixed lift of $P_m$ to the universal cover $\widetilde{K(n)}$. Note that the group $W_Q$ acts naturally on $\widetilde{K(n)}_0$ via translation, i.e. $w.(q,w')=(q,ww')$. Up to this action a lift $\tilde P_m$ of $P_m$ is unique. On the universal covering, every preprojective representation $\tilde P_m$ has $n$ subrepresentation $\tilde P_{m-1,1},\ldots,\tilde P_{m-1,n}$. Each of them corresponds (inductively) to one of the $n$ different arrows of $K(n)$. In turn, with every sequence of natural numbers $\uj=(j_1,\ldots,j_l)$ with $j_i\leq n$ and $l\leq m-1$ we can associate a preprojective subrepresentation $\tilde P_{m-k,\uj}$ of dimension $(u_{m-k},u_{m-k-1})$. For $\{i_1,\ldots,i_k\}\subset \{1,\ldots,n\}$ and $1\leq k\leq n-1$, there exist exact sequences of the form
\[\ses{\bigoplus_{j=1}^k\tilde P_{m-1,i_j}}{\tilde P_m}{\tilde P_{m}^{(i_1,\ldots,i_k)}},\quad\quad\ses{\tilde P_{m-1,i_k}}{\tilde P_{m}^{(i_1,\ldots,i_{k-1})}}{\tilde P_{m}^{(i_1,\ldots,i_k)}}.\]

Using this notation, we obtain the following lemma:
\begin{lemma}
  \label{le:preprojective lifts}
  \begin{enumerate}
    \item $\Hom_Q(P_{m-1},P_m)\cong\bigoplus_{i=1}^n \Hom_{\tilde Q}(\tilde P_{m-1,i},\tilde P_m)$, where each $\Hom_{\tilde Q}(\tilde P_{m-1,i},\tilde P_m)$ is one-dimensional;
   % \item For any proper subset $\{i_1,\ldots,i_k\}\subset\{1,\ldots,n\}$, there exists a short exact sequence
   %   \[0\longrightarrow \tilde P_{m-1,i_1}\oplus\cdots\oplus\tilde P_{m-1,i_k}\longrightarrow\tilde P_m\longrightarrow \tilde P_m^{i_1,\ldots,i_k}\longrightarrow 0;\]
    \item The representations $\tilde P_{m-1,i}$ are pairwise orthogonal;
    \item All nontrivial proper subrepresentations $U$ of $\tilde P_m^{(i_1,\ldots,i_k)}$ are preprojective.
  \end{enumerate}
\end{lemma}
\begin{proof}
The first statement should directly follow from \cite{gab} and the fact that $F_Q$ is a covering functor (precise reference). The second part follows for instance recursively by applying BGP-reflections.

For the third part, we proceed by induction. Clearly $\tilde P_m$ has only preprojective subrepresentations. We can assume without loss of generality that the subrepresentation $U$ is indecomposable. If $U\subsetneq P_m^{(i_1,\ldots,i_k)}$ were not preprojective, the embedding would not factor through $P_m^{(i_1,\ldots,i_{k-1})}$ by induction hypothesis.  Thus we get a commutative diagram

\[\xymatrix{0\ar[r] &\tilde P_{m-1,i_k}\ar[r] &  \tilde P_{m}^{(i_1,\ldots,i_{k-1})}\ar[r] & \tilde P_{m}^{(i_1,\ldots,i_{k})}\ar[r] & 0 \\
  0\ar[r]&\tilde P_{m-1,i_k} \ar[r]\ar[u] & U'\ar[u]\ar[r]& U\ar[r]\ar[u]& 0}\]
where the vertical maps are injective, the sequence on the bottom does not split and the representation $U'$ is preprojective by induction hypothesis. Then we have $U\in\tilde P_{m-1,i_k}^\perp$
because $$\Hom(\tilde P_{m-1,i_k},U)=\Hom(\tilde P_{m-1,i_k},\tilde P_{m}^{(i_1,\ldots,i_{k})})=0$$ and $\tilde P_{m-1,i_k}$ is preprojective and $U$ is not. Moreover, we have $\Hom(U,\tilde P_{m-1,i_k})=0$ for the same reason. Thus $U'$ is an indecomposable subrepresentation of $\tilde P_{m}^{(i_1,\ldots,i_{k})}$ (reference) such that
\[\Sc{\udim U'}{\udim U'}=\Sc{\udim U}{\udim U}+\Sc{\udim \tilde P_{m-1,i_k} }{\udim \tilde P_{m-1,i_k} }+\Sc{\udim U}{\tilde P_{m-1,i_k}}=1+\Sc{\udim U}{\tilde P_{m-1,i_k}}\leq 0\]
because we have $\Ext(U,\tilde P_{m-1,i_k})\neq 0$ as the bottom sequence does not split.
But this contradicts the induction hypothesis because preprojective representations do not have self-extensions.

\end{proof}
If we choose subset $\{i_1,\ldots,i_k\}=\{1,\ldots,k\}$, we denote the cokernel simply by $\tilde P_m^{(k)}$ and $\tilde P_m^{(0)}=\tilde P_m$ respectively. 

\begin{lemma}\label{directsums}
  If the quiver Grassmannians $Gr^{\tilde Q}_\tbfe(\tilde P_{m})$ and $Gr^{\tilde Q}_\tbfe(\tilde P_{m-1})$ have cell decompositions into affine spaces, then we have that $Gr^{\tilde Q}_\tbfe(\bigoplus_{j=1}^s\tilde P^{t_j}_{m-1,i_j}\oplus\bigoplus_{j=1}^k\tilde P^{l_j}_{m,i_j})$ has a cell decomposition into affine spaces where $l_j,t_j\geq 1$ and $s,k\leq n$.
\end{lemma}
\begin{proof}The claim follows by induction when considering the map $$\Psi:Gr^{\tilde Q}_\tbfe(\bigoplus_{j=1}^k\tilde P^{l_j}_{m,i_j})\to\bigsqcup_{\tbff+\tbfg=\tbfe}  Gr^{\tilde Q}_\tbff(\tilde P_{m,i_1})\times Gr^{\tilde Q}_\tbfg(\tilde P^{l_1-1}_{m,i_1}\oplus\bigoplus_{j=1}^k\tilde P^{l_j}_{m,i_j}),\,U\mapsto (U\cap \tilde P_{m-1,i_1},\pi(U))$$
which is also sometimes called Caldero-Chapoton map.

Indeed, we can use the fact that every subrepresentation $U$ of $\tilde P^{l_1-1}_{m,i_1}\oplus\bigoplus_{j=1}^k\tilde P^{l_j}_{m,i_j}$ is preprojective whence every proper factor $\tilde P_{m,i_1}/V$ is not which means that we have $\Ext(U,\tilde P_{m,i_1}/V)=0$. But since $U$ has only direct summands which are isomorphic to $\tilde P_{m-k}$ for some $k\geq 0$, we also have $\Ext(U,\tilde P_{m-1,i_1})=0$. Thus by the results of \cite[Section 3]{cc} it follows that $\psi^{-1}(U,V)=\mathbb{A}^{\Sc{\udim U}{\udim \tilde P_{m-1,i_1}/V}}$ for all pairs of subrepresentations $(U,V)\in Gr^{\tilde Q}_\tbff(\tilde P_{m,i_1})\times Gr^{\tilde Q}_\tbfg(\tilde P^{l_1-1}_{m,i_1}\oplus\bigoplus_{j=1}^k\tilde P^{l_j}_{m,i_j})$. In particular, the fibres of $\Psi$ are constant over every affine cell $A$ of the right hand side. Thus we can consider the restriction of $\Psi$ to $\Psi^{-1}(A)$. By \cite[Proposition 4]{cj}, it follows that this restriction is a subbundle of a trivial bundle (write this trivial bundle down) and thus it is itself trivial as a vector bundle over an affine space.

In the same manner, we can proceed by inductively adding the summands $\tilde P_{m-1,i}$ in order to see that $Gr^{\tilde Q}_\tbfe(\bigoplus_{j=1}^s\tilde P^{t_j}_{m-1,i_j}\oplus\bigoplus_{j=1}^k\tilde P^{l_j}_{m,i_j})$ has a cell decomposition.
\end{proof}

\begin{lemma}
  \begin{enumerate}
    \item $\tilde P_m^{(i_1,\ldots,i_{n-1})}\cong\tilde P_{m-1}^{(i_n)}$
    \item The subrepresentation $\bigoplus_{i=1}^{k-1}\tilde P_{m-1,i}\oplus\bigoplus_{i=1}^k\tilde P_{m-2,i}\subset\bigoplus_{i=1}^k\tilde P_{m-1,i}$ is in $(\tilde P_m^{(k)})^\perp$ and
      \[\Ext(\bigoplus_{i=1}^k\tilde P_{m-1,i},\tilde P_m^{(k)})\cong\Ext(\tilde P_{m-1}^{(k)},\tilde P_m^{(k)})\]
      where $\tilde P_{m-1}^{(k)}$ above denotes the cokernel of the inclusion.\footnote{do we need this? See also proof of Lemma 4.5}
  \end{enumerate}
\end{lemma}
\begin{proof}
The first statement follows for instance by induction when applying BGP-reflections.
\end{proof}
\begin{lemma}\label{AR}
Let $m\geq 3$ and $n\geq k\geq 1$. Consider the short exact sequence $\ses{\tilde P_{m,k}}{\tilde P_{m+1}^{(k-1)}}{\tilde P_{m+1}^{(k)}}$. 
\begin{enumerate}
\item There exists a short exact sequence 
\[\ses{\bigoplus_{\substack{1\leq j\leq n\\j\neq k}}\tilde P_{m-1,(k,j)}\oplus \bigoplus_{j=1}^k\tilde P_{m-2,(k,k,j)}}{\tilde P_{m,k}}{\tau\tilde P_{m+1}^{(k)}}\]
where $\tau$ denotes the Auslander-Reiten translate where
$\tau\tilde P_{m+1}^{(k)}=\tilde P_{m-1}^{(k)}$. 
\item Moreover, we have $\Ext(\tilde P_{m+1}^{(k)},\tilde P_{m,k})\cong \Ext(\tilde P_{m+1}^{(k)},\tau\tilde P_{m+1}^{(k)})=\kk$, $\Hom(\tilde P_{m+1}^{(k)},\tau\tilde P_{m+1}^{(k)})=0$. 
\item We have $\tau\tilde P_{m+1}^{(k)}\in ~^\perp(\tilde P_{m+1}^{(k)})$.
%\item Finally, we have $\Ext(\tilde P_{m+1}^{(k-1)},V)=0$ for every subrepresentation $V$ of $\tilde P_{m-1}^{n-1}\oplus \tilde P_{m-2}^{n-k}$.
\end{enumerate}%\footnote{introduce/consider/check indices, we have to mod out the corresponding reps in the covering} 
\end{lemma}
\begin{proof}
We have $\mathcal  S \tilde P_m=\tilde P_{m+1}$ (up to permutation). This means that it is enough to show the statement for $m=3$ as all representations appearing are not isomorphic to the simple injective representation $S_2$. In this case, the first statement is straightforward as well as the third one (insert a diagram).

It is also straightforward to check that $\bigoplus_{j\neq k}\tilde P_{2,(k,j)}\oplus \bigoplus_{j=1}^k\tilde P_{1,(k,k,j)}\in (\tilde P_{4}^{(k)})^\perp$ and, moreover, that $\Hom(\tilde P_{4}^{(k)},\tilde P_{3,k})=0$ which yields the second claim when considering the long exact sequence obtained when applying $\Hom(\tilde P_{4}^{(k)},\underline{\quad})$.


\end{proof}
The short exact sequence $\ses{\tilde P_{m,k}}{\tilde P_{m+1}^{(k-1)}}{\tilde P_{m+1}^{(k)}}$ induces a map between quiver Grassmannians
$$\Psi_1:Gr^{\tilde Q}_\tbfe(\tilde P_{m+1}^{(k-1)})\to\bigsqcup_{\tbff+\tbfg=\tbfe} Gr^{\tilde Q}_\tbff( \tilde P_{m,k})\times Gr^{\tilde Q}_\tbfg(\tilde P_{m+1}^{(k)}).$$
Let $P(m,k):=\bigoplus_{\substack{1\leq j\leq n\\j\neq k}}\tilde P_{m-1,(k,j)}\oplus \bigoplus_{j=1}^k\tilde P_{m-2,(k,k,j)}$.

By Lemma \ref{AR}, every $U\subset \tilde P_{m,k}$ gives rise to a commutative diagram
 
 \[\xymatrix{0\ar[r] &P(m,k)\ar[r] &  \tilde P_{m,k}\ar[r] & \tau\tilde P_{m+1}^{(k)}\ar[r] & 0 \\
  0\ar[r]& V\ar[r]\ar[u] & U\ar[u]\ar[r]& W\ar[r]\ar[u]& 0}\]
By $\Psi_2$ we denote the corresponding map of quiver Grassmannians.
Using this notation, we obtain:
\begin{lemma}\label{quotient}
\begin{enumerate}
\item The fibre $\Psi_1^{-1}(U,\tilde P_{m+1}^{(k)})$ is empty if and only if $U\in\Psi^{-1}_2(V,0)$, i.e. $U$ is already a subrepresentation of $P(m,k)$.
\item If the fibre $\Psi_1^{-1}(U,\tilde P_{m+1}^{(k)})$ is not empty, we have $\Ext(\tilde P_{m+1}^{(k)},\tilde P_{m,k}/U)= 0$.
\end{enumerate} 
\end{lemma}
\begin{proof}
If $W=0$, we get an exact sequence
\[\ses{\Ext(\tilde P_{m+1}^{(k)},P(m,k)/V)}{\Ext(\tilde P_{m+1}^{(k)},\tilde P_{m,k}/U)}{\Ext(\tilde P_{m+1}^{(k)},\tau\tilde P_{m+1}^{(k)})}.\]

By Lemma \ref{AR}, we have $\Hom(\tilde P_{m+1}^{(k)},\tau \tilde P_{m+1}^{(k)})=0$, $\Ext(\tilde P_{m+1}^{(k)},\tau \tilde P_{m+1}^{(k)})\cong\Ext(\tilde P_{m+1}^{(k)},\tilde P_{m,k})=\kk$ and $\Hom(\tilde P_{m+1}^{(k)},\tilde P_{m,k})=0$ and thus $P(m,k)\in (\tilde P_{m+1}^{(k)})^\perp$. Since this means $\Ext(\tilde P_{m+1}^{(k)},P(m,k))=0$, we get $\Ext(\tilde P_{m+1}^{(k)},P(m,k)/V)=0$. This yields $\Ext(\tilde P_{m+1}^{(k)},\tilde P_{m,k}/U)\cong\Ext(\tilde P_{m+1}^{(k)},\tau\tilde P_{m,k})=\kk$ which means that the induced map
\[\Ext(\tilde P_{m+1}^{(k)},U)\to\Ext(\tilde P_{m+1}^{(k)},\tilde P_{m,k})\]
is zero. Therefore, the induced exact sequence splits. On the other hand, every subrepresentation $U\subset \tilde P_{m,k}$ such that $\Psi_1^{-1}(U,\tilde P_{m+1}^{(k)})$ is not empty comes with a commutative diagram
 \[\xymatrix{0\ar[r] &\tilde P_{m,k}\ar[r] &  \tilde P_{m+1}^{(k-1)}\ar[r] & \tilde P_{m+1}^{(k)}\ar[r] & 0 \\
  0\ar[r]& U\ar[r]\ar[u] & U'\ar[u]\ar[r]& \tilde P_{m+1}^{(k)}\ar[r]\ar[u]& 0}\]
which means that the induced sequence in $\Ext(\tilde P_{m+1}^{(k)},\tilde P_{m,k})$ does not split. This yields that the fibre is empty.


Thus assume that $U\in\Psi_2^{-1}(V,W)$ with $W\neq 0$. We first show that $\Ext(\tilde P_{m+1}^{(k)},\tilde P_{m,k}/U)= 0$. Clearly, $\tau\tilde P_{m+1}^{(k)}/W$ is a proper factor of $\tau \tilde P_{m+1}^{(k)}$ if $W\neq 0$. 
It follows that we have $\Ext(\tilde P_{m+1}^{(k)},\tau\tilde P_{m+1}^{(k)}/W)=0$. Indeed, by Auslander-Reiten theory every non-split morphism $g:\tau\tilde P_{m+1}^{(k)}\to \tau\tilde P_{m+1}^{(k)}/W$ factors through the middle term $Z$ of the AR-sequence 
\[\ses{\tau\tilde P_{m+1}^{(k)}}{Z}{\tilde P_{m+1}^{(k)}}\]
which means that the first map of the induced sequence
\[\Hom(\tau\tilde P_{m+1}^{(k)},\tau\tilde P_{m+1}^{(k)}/W)\to\Ext(\tilde P_{m+1}^{(k)},\tau\tilde P_{m+1}^{(k)}/W)\to\Ext(Z,\tau\tilde P_{m+1}^{(k)}/W) \]
is zero. As $\tau\tilde P_{m+1}^{(k)}$ is exceptional (on the universal cover), we are in the situation of Ringel's reflection \cite{rin} which means that $\Ext(\tilde P_{m+1}^{(k)},\tau\tilde P_{m+1}^{(k)})=\kk$, $\Hom(\tilde P_{m+1}^{(k)},\tau\tilde P_{m+1}^{(k)})=0$ and $\tau\tilde P_{m+1}^{(k)}\in ~^\perp(\tilde P_{m+1}^{(k)})$, i.e. Lemma \ref{AR}, yields $\Ext(Z,\tau\tilde P_{m+1}^{(k)})=0$ as the AR-sequence does not split. In conclusion, we obtain $\Ext(\tilde P_{m+1}^{(k)},\tau\tilde P_{m+1}^{(k)}/W)=0$. 


Again we can apply Lemma \ref{AR} in order to see that $\Ext(\tilde P_{m+1}^{(k)},P(m,k)/V)=0$ as $\Ext(\tilde P_{m+1}^{(k)},P(m,k))=0$. 
 Together with $\Ext(\tilde P_{m+1}^{(k)},\tau\tilde P_{m+1}^{(k)}/W)=0$, this yields the claim.

Thus it remains to show that $\Psi^{-1}_1(U,\tilde P_{m+1}^{(k)})$ is not empty if $W\neq 0$. Since $\Ext(\tilde P_{m+1}^{(k)},\tilde P_{m,k}/U)= 0$, we can conclude that the map $\Ext(\tilde P_{m+1}^{(k)},U)\to\Ext(\tilde P_{m+1}^{(k)},\tilde P_{m,k})$ is surjective which already means that there exists a commutative diagram as above which, in turn, means that the fiber is not empty.
\end{proof}
Now we are able to state the following result concerning the fibers of $\Psi_1$:
\begin{proposition}\label{fibers}\footnote{maybe formulate this in terms of $\Psi$ (see below)}
 The following hold:
  \begin{enumerate}
    \item For $V\subsetneq \tilde P_{m+1}^{(k)}$ and $U\subseteq\tilde P_{m,k}$, we have $\Psi_1^{-1}(U,V)=\AA^{\langle V,\tilde P_{m,k}/U\rangle}$.
    \item If $V=\tilde P_{m+1}^{(k)}$, the fiber $\Psi_1^{-1}(U,V)$ is not empty if and only if $\Psi_2(U)\neq (U,0)$. In this case, we have $\Psi_1^{-1}(U,V)=\AA^{\langle V,\tilde P_{m,k}/U\rangle}$.
		
  \end{enumerate}
\end{proposition}
\begin{proof}By Lemma \ref{le:preprojective lifts}.3, any subrepresentation $V\subsetneq \tilde P_{m+1}^{(k)}$ is preprojective. But the representation $\tilde P_{m,k}/U$ is not as it is a proper quotient of a preprojective representation unless $U=0$. Thus we have $\Ext(V,\tilde P_{m,k}/U)=0$. If $U=0$, we have $\Ext(V,\tilde P_{m,k})=0$ because for dimension reasons every indecomposable direct summand of $V$ is isomorphic to some $\tilde P_l$ with $l\leq m$.

The second statement follows directly from Lemma \ref{quotient}.

\end{proof}
\begin{theorem}\label{cellscover}
Every quiver Grassmannian $Gr^{\tilde Q}_\tbfe(\tilde P_{m+1}^{(k-1)})$ admits a cell decomposition into affine spaces.
\end{theorem}
\begin{proof}
We proceed by induction and combine the two maps $\Psi_1$ and $\Psi_2$ and obtain a map
\[\Psi:Gr^{\tilde Q}_\tbfe(\tilde P_{m+1}^{(k-1)})\to\bigsqcup_{\tbff+\tbfg+\tbfh=\tbfe} Gr^{\tilde Q}_\tbff( P(m,k))\times Gr^{\tilde Q}_\tbfg(\tau\tilde P_{m+1}^{(k)})\times Gr^{\tilde Q}_\tbfh(\tilde P_{m+1}^{(k)}).\]

By Lemma \ref{directsums} and induction hypothesis every quiver Grassmannian on the right hand side has a cell decomposition. The second part of the Proposition says that the fibres are empty if and only if $\tbfg=0$ and $\tbfh=\udim \tilde P_{m+1}^{(k)}$. In all other cases, the fiber dimension only depend on the dimension vectors $\tbff, \tbfg,\tbfh$ which means that they are constant over $Gr^{\tilde Q}_\tbff( P(m,k))\times Gr^{\tilde Q}_\tbfg(\tau\tilde P_{m+1}^{(k)})\times Gr^{\tilde Q}_\tbfh(\tilde P_{m+1}^{(k)})$\footnote{actually this should already say that this is a vector bundle as the base space is smooth (all reps are exceptional)}. Analogously to Lemma \ref{directsums}, we can conclude that this induces trivial bundles over every affine cell. In total, this yields a cell decomposition of the quiver Grassmannian $Gr^{\tilde Q}_\tbfe(\tilde P_{m+1}^{(k-1)})$.

\end{proof}


Note that we can also consider 
$\Psi_1:Gr^{\tilde Q}_\tbfe(\tilde P^{(k-1)}_{m})\to\bigsqcup_{\tbff+\tbfg=\tbfe} Gr^{\tilde Q}_\tbff(\bigoplus_{i=1}^{k}\tilde P_{m-1,i})\times Gr^{\tilde Q}_\tbfg(\tilde P_m^{(k)})$ where $\tilde P_m^{(n-1)}=\tilde P_{m-1}^{(1)}$ (up to renumbering the arrows). But then one needs to adjust $\Psi_2$ because the factor of $\bigoplus_{i=1}^{k}\tilde P_{m-1,i}$ were $(\tau\tilde P_m^{(k)})^k$.


\begin{theorem}
  Every quiver Grassmannian of a preprojective or preinjective representation of $K(n)$ and $\widetilde{K(n)}$ has a cell decomposition.
\end{theorem}
\begin{proof}
This follows by combining the results of Section 2 and Proposition \ref{fibers}.
\end{proof}
\begin{question}
  cells of $Gr^{\tilde Q}_\tbfe(\tilde P_m)$ are in one-to-one correspondence with certain tuples of subgraphs for smaller $\tilde P_\ell^{i_1,\ldots,i_k}$
\end{question}


%%%%%%%%%%%%%%%%%%%%%%%%%%%%%%%%%%%%%%%%%%%%%%%%%%%%%%%%%%
\section{Compatible Pairs Label Cells in $Gr^Q_\bfe(P_m)$}
\begin{thebibliography}{10}
\bibitem{ass}
Assem, I., Simson, D., Skowronski, A.: Elements of the Representation Theory of Associative Algebras. Cambridge University Press, Cambridge 2007.
\bibitem{bb} Bialynicki-Birula, A.: Some theorems on actions of algebraic groups. Annals of Mathematics \textbf{98}, 480-497 (1973).
\bibitem{bgp}
Bernstein, I.~N., Gelfand, I.~M., Ponomarev, V.~A.: Coxeter functors, and Gabriel's theorem. Russian Mathematical Surveys \textbf{28}(2), 17-32 (1973).
\bibitem{cc}
Caldero, P., Chapoton, F.: Cluster algebras as {H}all algebras of quiver representations.
Commentarii Mathematici Helvetici \textbf{81}(3), 595-616 (2006).
\bibitem{cr}
Caldero, P., Reineke, M.: On the quiver Grassmannian in the acyclic case.
Journal of Pure and Applied Algebra \textbf{212}(11), 2369-2380 (2008).
\bibitem{cj} Crawley-Boevey, W., Jensen, Bernt Tore: A note on sub-bundles of vector bundles. Glasgow Mathematical Journal \textbf{48}, 459-462 (2006).

\bibitem{gab} Gabriel, P.: The universal cover of a finite-dimensional algebra. Representations of algebras. Lecture Notes in Mathematics {\bf 903}, 68-105 (1981).
\bibitem{hr} Happel, D., Ringel, C.M.: Tilted Algebras. Transactions of the American Mathematical Society {\bf 274}, no.2, 399-443 (1982).
\bibitem{rin} Ringel, C.M.: Reflection functors for hereditary algebras. Journal of the London Mathematical Society (2) {\bf 21}, no. 3, 465-479 (1980).

\bibitem{sch} Schofield, A.: General representations of quivers. Proceedings of the London Mathematical Society (3) \textbf{65}(1), 46-64 (1992).
	\bibitem{wei} Weist, T.: Localization of quiver moduli spaces. Representation Theory \textbf{17}(13), 382-425 (2013).
\end{thebibliography}

\end{document}
