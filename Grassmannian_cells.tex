\documentclass{amsart}
\usepackage{amsmath,amssymb,latexsym}
\usepackage[margin=1in]{geometry}
\usepackage{extarrows}
\input xy
\xyoption{all}

\newtheorem{theorem}{Theorem}[section]
\newtheorem{corollary}[theorem]{Corollary}
\newtheorem{definition}[theorem]{Definition}
\newtheorem{lemma}[theorem]{Lemma}
\newtheorem{question}[theorem]{Question}
\newtheorem{proposition}[theorem]{Proposition}

\newcommand{\bfe}{\mathbf{e}}
\newcommand{\bff}{\mathbf{f}}
\newcommand{\bfg}{\mathbf{g}}
\newcommand{\tbfe}{{\tilde\bfe}}
\newcommand{\tbff}{{\tilde\bff}}
\newcommand{\tbfg}{{\tilde\bfg}}
\newcommand{\C}{\mathbb{C}}

\newcommand{\Rep}{\mathrm{Rep}}
\newcommand{\Gr}{\mathrm{Gr}}
\renewcommand{\AA}{\mathbb{A}}
\newcommand{\kk}{\Bbbk}
\newcommand{\ZZ}{\mathbb{Z}}
\newcommand{\Ext}{\operatorname{Ext}}
\newcommand{\Hom}{\operatorname{Hom}}
\newcommand{\ses}[3]{0\rightarrow #1\rightarrow #2\rightarrow#3\rightarrow 0}

\title{Cell Decomposition of Rank 2 Quiver Grassmannians}

\author{Dylan Rupel}
\author{Thorsten Weist}

\begin{document}
\begin{abstract}
  We prove that all quiver Grassmannians for preprojective and preinjective representations of a generalized Kronecker quiver admit a cell decomposition.  
  We also provide a natural combinatorial labeling for these cells using compatible pairs in a maximal Dyck path. 
\end{abstract}
\maketitle

\section{Introduction}
-something about cluster algebras
-something about categorification and quiver Grassmannians
-something about compatible pairs and combinatorial construction of cluster variables
-statement of our results
-something about torus actions and the universal cover
-something about cell decompositions of quiver Grassmannians
-acknowledgements?


%%%%%%%%%%%%%%%%%%%%%%%%%%%%%%%%%%%%%%%%%%%%%%
\section{Torus Action on Quiver Grassmannians}
\noindent Let $Q$ be an acylic quiver with vertices $Q_0$ and arrows $Q_1$ which we denote by $\alpha:i\to j$. Moreover, let $W_{Q}$ be the free (non-abelian) group generated by $Q_1$.  We denote by $\Rep(Q)$ the category of $\C$-representations of $Q$.
\begin{definition}
The universal covering quiver $\tilde Q$ of $Q$ is given by the vertices $\tilde Q_0=Q_0\times W_{Q}$ and the arrow set $\tilde Q_1=Q_1\times W_{Q}$ where $(\alpha,w):(i,w)\to (j,w\alpha)$ for every $\alpha:i\to j$.

We say that a representation $X\in \Rep(Q)$ can be lifted (to $\tilde Q$) if there exists a representation $\tilde X\in\Rep(\tilde Q)$ such that $F_Q\tilde X=X$ where $F_Q:\Rep(\tilde Q)\to\Rep(Q)$ is the natural functor.
\end{definition}
\begin{lemma}
  Every preprojective (resp. preinjective) representation of $Q$ can be lifted to a representation of $\tilde Q$.
\end{lemma}
\begin{proof}This statement is clear for the simple representations $S_q$, $q\in Q_0$. Now every preprojective representation $X$ can be obtained when applying a sequence of BGP-reflections \cite{bgp} to a simple representation $S_{q'}$ of a quiver $Q'$ whose underlying graph is the one of $Q$. Applying BGP-reflections to a source $q$ of $Q$ corresponds to applying BGP-reflections to all vertices $(q,w)$ of $\tilde Q$. This leads the claim. The statement for preinjective representations follows in the same way.
\end{proof}
We choose a map $d:\tilde Q_0\to\ZZ$ and fix a representation $X\in\Rep(Q)$. In any case, we can consider the decomposition $X_q=\bigoplus_{w\in W_Q} X_{(q,w)}$. We define a torus action on each $X_{(q,w)}$ via $t.x_{(q,w)}=t^{d(q,w)}x_{(q,w)}$ which can be extended linearly to each $X_q$. For a fixed a subspace $U_q$, we can define the subspace $t.U_q$. In general, this torus action induces no torus action on the Quiver Grassmannians $\Gr_{\bfe}(X)$ as $t.U=(t.U_q)_{q\in Q_0}$ is no subrepresentation of $X$ for every $U\in \Gr_{\bfe}(X)$. Actually, for this the action has to satisfy $X_{\alpha}(t.U_i)\in t.U_j$ for every $\alpha:i\to j$ and every $x\in X_i$. 
\begin{lemma}Fix an integer $c_\alpha\in\ZZ$ for every $\alpha\in Q_1$.
If $X\in\Rep(Q)$ can be lifted, $d:\tilde Q_0\to\ZZ$ induces a torus action on $\Gr_\bfe(X)$ if we have $d(j,w\alpha)-d(i,w)=c_\alpha$ for all $\alpha:i\to j$ and $w\in W_{Q}$.
\end{lemma}
\begin{proof} Since $X$ can be lifted, we can write $X_\alpha:X_i\to X_j$ as block matrix consisting of linear maps $X_{(\alpha,w)}:X_{(i,w)}\to X_{(j,w\alpha)}$. Then the condition $X_{\alpha}(t.U_i)\in t.U_j$ translates into ...to be continued.
\end{proof}

\begin{lemma}
  There exists $d:\tilde Q_0\to\ZZ$ such that $d(q,w)\ne d(q',w')$ for all $q,q',w,w'$ with $\dim X_{q,w}\ne0$ and $\dim X_{q',w'}\ne0$.
\end{lemma}

\begin{theorem}
  $Gr^Q_\bfe(X)^T\cong\bigsqcup_{\tilde\bfe of type \bfe} Gr^{\tilde Q}_{\tilde\bfe}(\tilde X)$
\end{theorem}

\begin{corollary}
  affine bundles over $Gr^{\tilde Q}_{\tilde\bfe}(\tilde X)$ if $Gr^Q_\bfe(X)$ is smooth
  \[\{U\in Gr^Q_\bfe(X):\lim_{t\to0} t\cdot U\in Gr^{\tilde Q}_{\tilde\bfe}(\tilde X)\}\]
\end{corollary}

\begin{question}
  What are the ranks of these bundles?  Poincar\'e polynomials?
\end{question}


%%%%%%%%%%%%%%%%%%%%%%%%%%%%%%%%%%%%%%%%%
\section{Representation Theory of $K(n)$}

Denote by $K(n)$ the \emph{$n$-Kronecker quiver} $1\stackrel{n}{\longleftarrow}2$ with vertices $K(n)_0=\{1,2\}$ and $n$ arrows from vertex $2$ to vertex $1$.  

Define Chebyshev polynomials $u_k$ for $k\in\ZZ$ by the recursion $u_0=0$, $u_1=1$, $u_{k+1}=nu_k-u_{k-1}$.
\begin{theorem}
  For each $m\ge1$, there exist unique (up to isomorphism) indecomposable rigid representations $P_m$ and $I_m$ of $K(n)$ with dimension vectors $(u_m,u_{m-1})$ and $(u_{m-1},u_m)$ respectively.
  Moreover, any rigid representation of $K(n)$ is isomorphic to one of the form $P_m^{a_1}P_{m+1}^{a_2}$ or $I_m^{a_1}I_{m+1}^{a_2}$ for some $m\ge1$ and some $a_1,a_2\ge0$.
\end{theorem}
The representations $P_m$ are called the \emph{preprojective} representations of $K(n)$ and the representations $I_m$ are called \emph{preinjective}.

\begin{lemma}
  For any $m\ge1$, we have 
  \[\dim\Hom(P_m,P_m)=1,\quad \dim\Hom(P_m,P_{m+1})=n,\quad \dim\Hom(P_{m+1},P_m)=0.\]  
  Moreover, for $1\le k<n$ and linearly independent functions $f_1,\ldots,f_k\in\Hom(P_m,P_{m+1})$, the map $[f_1\ \cdots\ f_k]:P_m^k\to P_{m+1}$ is injective. 
\end{lemma}
\begin{proof} 
  Note that we have an Auslander-Reiten sequence
  \begin{equation}
  \label{eq:AR sequence}
    0\longrightarrow P_{m-1}\longrightarrow P_m^n\stackrel{[f_1\ \cdots\ f_n]}{\longrightarrow} P_{m+1}\longrightarrow 0
  \end{equation}
  with irreducible homomorphisms $f_i$ which are also linearly independent. 
  Thus they do not factor through a representation $Z\neq P_{m}$ which already means that they are injective, see for instance \cite[Lemma 1.6]{ass}. 
  If we pick $k<n$ linear independent homomorphisms $f_{i_1},\ldots, f_{i_k}$, this also means that $[f_{i_1}\ \cdots\ f_{i_k}]:P_m^k\to P_{m+1}$ is injective. 
  Indeed otherwise, at least one of the homomorphisms were forced to factor through a representation $Z\neq P_{m}$.
\end{proof}

Write $P_{m+1}^{\{f_1,\ldots,f_k\}}$ for the cokernel of the map $[f_1\ \cdots\ f_k]:P_m^k\to P_{m+1}$, i.e. we have a short exact sequence
\begin{equation}
\label{eq:truncated preprojectives}
  0\longrightarrow P_m^k\stackrel{[f_1\ \cdots\ f_k]}{\longrightarrow} P_{m+1}\longrightarrow P_{m+1}^{\{f_1,\ldots,f_k\}}\longrightarrow 0.
\end{equation}

\begin{lemma}
  For $m\ge1$ and linearly independent functions $f_1,\ldots,f_k\in\Hom(P_m,P_{m+1})$, the following hold:
  \begin{enumerate}
    \item $\Hom(P_m,P_{m+1}^{\{f_1,\ldots,f_k\}})$ is naturally isomorphic to $\Hom(P_m,P_{m+1})/\kk f_1\oplus\cdots\oplus\kk f_k$;
    \item $\Ext(P_{m+1}^{\{f_1,\ldots,f_k\}},P_m)$ is naturally isomorphic to $\kk f_1\oplus\cdots\oplus\kk f_k$;
    \item $\Hom(P_{m+1}^{\{f_1,\ldots,f_k\}},P_m)=0$.
  \end{enumerate}
\end{lemma}
\begin{proof}
  Applying the functor $\Hom(P_m,-)$ to the sequence \eqref{eq:truncated preprojectives} gives the exact sequence
  \[0\longrightarrow \Hom(P_m,P_m^k)\stackrel{[f_1\ \cdots\ f_k]\circ}{\longrightarrow} \Hom(P_m,P_{m+1})\longrightarrow \Hom(P_m,P_{m+1}^{\{f_1,\ldots,f_k\}})\longrightarrow 0,\]
  establishing (1).  
  
  To see (2), we apply the functor $\Hom(-,P_m)$ to the sequence \eqref{eq:truncated preprojectives} to get an isomorphism $\Hom(P_m^k,P_m)\cong\Ext(P_{m+1}^{\{f_1,\ldots,f_k\}},P_m)$.
  Observe that $\Hom(P_m^k,P_m)$ is spanned by the coordinate projections $\pi_i:P_m^k\to P_m$ for $1\le i\le k$ and each of these gives rise to a commutative diagram
  \[\xymatrix{ & 0\ar[d] & 0\ar[d] & & \\ & P_m^{k-1}\ar[d]\ar@{=}[r]& P_m^{k-1}\ar[d]^{[f_1\ \cdots\ \widehat{f_i}\ \cdots\ f_k]}& & \\
    0\ar[r]& P_m^k\ar[r]^{[f_1\ \cdots\ f_k]}\ar[d]^{\pi_i}\ar@{}[dr]|(.35){\ulcorner}& P_{m+1}\ar[r]\ar[d]& P_{m+1}^{\{f_1,\ldots,f_k\}}\ar[r]\ar@{=}[d]& 0 \\
    0\ar[r]& P_m\ar[r]^(.35){f_i}\ar[d]& P_{m+1}^{\{f_1,\ldots,\widehat{f_i},\ldots,f_k\}}\ar[r]\ar[d]& P_{m+1}^{\{f_1,\ldots,f_k\}}\ar[r]& 0\\
     & 0 & 0 & & }\]
  The bottom row of this diagram is the extension in $\Ext(P_{m+1}^{\{f_1,\ldots,f_k\}},P_m)$ naturally identified with the generator $f_i$ of $\kk f_1\oplus\cdots\oplus\kk f_k$.

  Part (3) follows immediately from (2) and the observation that 
  \[\langle P_{m+1}^{\{f_1,\ldots,f_k\}},P_m\rangle=\langle P_{m+1},P_m\rangle-k\langle P_m,P_m\rangle=-k.\]
\end{proof}

\begin{lemma}
  Let $f_1,\ldots,f_n\in\Hom(P_m,P_{m+1})$ be linearly independent morphisms and write $g:P_{m-1}\to P_m$ for the composition of the inclusion from \eqref{eq:AR sequence} with the projection $\pi_n:P_m^n\to P_m$ to the $n^{th}$ factor.  Then $P_{m+1}^{\{f_1,\ldots,f_{n-1}\}}\cong P_m^{\{g\}}$.
\end{lemma}
\begin{proof}
  We have the following commutative diagram
  \[\xymatrix{0\ar[r]& P_m^{n-1}\ar@{=}[d]\ar[r]& P_m^n\ar[r]^{\pi_n}\ar[d]^{[f_1\ \cdots\ f_n]}\ar@{}[dr]|(.65){\lrcorner}& P_m\ar[r]\ar[d]^{f_n}& 0 \\
  0\ar[r]& P_m^{n-1}\ar[r]^{[f_1\ \cdots\ f_{n-1}]}& P_{m+1}\ar[r]& P_{m+1}^{\{f_1,\ldots,f_{n-1}\}}\ar[r]& 0}\]
  Since the vertical morphism on the left is an equality, the right hand square is a pullback.  
  But this implies the kernels of the vertical morphisms coincide.  
  The middle vertical morphism fits into the exact sequence \eqref{eq:AR sequence} and so the kernel is $P_{m-1}$.
  It follows that there is a morphism $g:P_{m-1}\to P_m$ so that $P_{m+1}^{\{f_1,\ldots,f_{n-1}\}}\cong P_m^{\{g\}}$.
\end{proof}

For any nonempty subset $I\subset[1,k]$, say with $|I|=\ell$, we get an exact sequence
\[0\longrightarrow P_m^\ell\stackrel{[f_i]_{i\in I}}{\longrightarrow} P_{m+1}^{\{f_j\}_{j\in[1,k]\setminus I}}\longrightarrow P_{m+1}^{\{f_1,\ldots,f_k\}}\longrightarrow 0.\]
Each such sequence has the following almost-split property for subrepresentations of $P_{m+1}^{\{f_1,\ldots,f_k\}}$.
\begin{lemma}
  For $m\ge1$ and linearly independent functions $f_1,\ldots,f_k\in\Hom(P_m,P_{m+1})$, given any proper subrepresentation $V\subsetneq P_{m+1}^{\{f_1,\ldots,f_k\}}$, the upper pullback sequence
  \[\xymatrix{0\ar[r] & P_m^\ell\ar@{=}[d]\ar[r] & V^I\ar[r]\ar[d]\ar@{}[dr]|{\lrcorner} & V\ar[d]\ar[r] & 0 \\
  0\ar[r]& P_m^\ell\ar[r]^(.35){[f_i]_{i\in I}} & P_{m+1}^{\{f_j\}_{j\in[1,k]\setminus I}}\ar[r]& P_{m+1}^{\{f_1,\ldots,f_k\}}\ar[r]& 0}\]
  splits for every $I\subset[1,k]$.
\end{lemma}
\begin{proof}
  When $I=\emptyset$, we have $\ell=0$ and there is nothing to show, so assume $I$ is a nontrivial subset of $[1,k]$.  
  
  We proceed by simultaneous induction on $k$ and $\ell=|I|$.  For any nonzero $f\in\Hom(P_m,P_{m+1})$ the exact sequence
  \[\xymatrix{0\ar[r] & P_m\ar[r]^f & P_{m+1}\ar[r] & P_{m+1}^{\{f\}}\ar[r] & 0}\]
  is almost split, giving the claim in the case $k=\ell=1$.  For $k>1$ and a nontrivial proper subset $I\subsetneq[1,k]$, we have the following commutative diagram
  \[\xymatrix{ & & 0\ar[d] & 0\ar[d] & \\   & 0\ar[r]\ar[d] & P_m^{k-\ell}\ar@{=}[r]\ar[d]^{[f_j]_{j\in[1,k]\setminus I}} & P_m^{k-\ell}\ar[d]^{[f_j]_{j\in[1,k]\setminus I}}\ar[r] & 0 \\
  0\ar[r] & P_m^\ell\ar@{=}[d]\ar[r]^{[f_i]_{i\in I}} & P_{m+1}\ar[r]\ar[d]\ar@{}[dr]|{\lrcorner} & P_{m+1}^{\{f_i\}_{i\in I}}\ar[d]\ar[r] & 0 \\
  0\ar[r]& P_m^\ell\ar[r]^(.35){[f_i]_{i\in I}}\ar[d] & P_{m+1}^{\{f_j\}_{j\in[1,k]\setminus I}}\ar[r]\ar[d] & P_{m+1}^{\{f_1,\ldots,f_k\}}\ar[r]\ar[d]& 0\\ & 0 & 0 & 0 & }\]
  A proper subrepresentation $V\subsetneq P_{m+1}^{\{f_1,\ldots,f_k\}}$ gives rise, via pullbacks, to the following commutative diagram
  \[\xymatrix{ & & 0\ar[d] & 0\ar[d] & \\   & 0\ar[r]\ar[d] & P_m^{k-\ell}\ar@{=}[r]\ar[d]^{[f_j]_{j\in[1,k]\setminus I}} & P_m^{k-\ell}\ar[d]^{[f_j]_{j\in[1,k]\setminus I}}\ar[r] & 0 \\
  0\ar[r] & P_m^\ell\ar@{=}[d]\ar[r]^{[f_i]_{i\in I}} & V^{[1,k]}\ar[r]\ar[d]\ar@{}[dr]|{\lrcorner} & V^{[1,k]\setminus I}\ar[d]\ar[r] & 0 \\
  0\ar[r]& P_m^\ell\ar[r]^(.35){[f_i]_{i\in I}}\ar[d] & V^I\ar[r]\ar[d] & V\ar[r]\ar[d]& 0\\ & 0 & 0 & 0 & }\]
  in which the middle vertical sequence and the middle horizontal sequence split by induction.  
  But the images of $[f_i]_{i\in I}$ and $[f_j]_{j\in[1,k]\setminus I}$ intersect trivially inside $V^{[1,k]}$ and thus the lower horizontal sequence must split as well, establishing the claim for $I\subsetneq[1,k]$.

  For $I=[1,k]$, we consider the commutative diagram
  \[\xymatrix{ & & 0\ar[d] & 0\ar[d] & \\   & 0\ar[r]\ar[d] & P_{m+1}\ar@{=}[r]\ar[d] & P_{m+1}\ar[d]\ar[r] & 0 \\
  0\ar[r] & P_m^{k-1}\oplus P_m\ar@{=}[d]\ar[r] & P_{m+1}\oplus P_{m+1}\ar[r]\ar[d]\ar@{}[dr]|{\lrcorner} & P_{m+1}^{\{f_1\ \cdots\ f_{k-1}\}}\oplus P_{m+1}^{\{f_k\}}\ar[d]\ar[r] & 0 \\
  0\ar[r]& P_m^k\ar[r]^{[f_1\ \cdots\ f_k]}\ar[d] & P_{m+1}\ar[r]\ar[d] & P_{m+1}^{\{f_1,\ldots,f_k\}}\ar[r]\ar[d]& 0\\ & 0 & 0 & 0 & }\]
  A proper subrepresentation $V\subsetneq P_{m+1}^{\{f_1,\ldots,f_k\}}$ then gives rise to the following commutative diagram
  \[\xymatrix{ & & 0\ar[d] & 0\ar[d] & \\   & 0\ar[r]\ar[d] & V^{[1,k]}\ar@{=}[r]\ar[d]^{\Delta} & V^{[1,k]}\ar[d]\ar[r] & 0 \\
  0\ar[r] & P_m^{k-1}\oplus P_m\ar@{=}[d]\ar[r] & V^{[1,k]}\oplus V^{[1,k]}\ar[r]\ar[d]\ar@{}[dr]|{\lrcorner} & V^{[1,k]\setminus I}\oplus V^I\ar[d]\ar[r] & 0 \\
  0\ar[r]& P_m^k\ar[r]\ar[d] & V^{[1,k]}\ar[r]\ar[d] & V\ar[r]\ar[d]& 0\\ & 0 & 0 & 0 & }\]
  in which the middle vertical sequence splits and the middle horizontal sequence is split by induction.  
  But the image of $[f_1\ \cdots\ f_k]:P_m^{k-1}\oplus P_m\to V^{[1,k]}\oplus V^{[1,k]}$ and the diagonal $\Delta:V^{[1,k]}\to V^{[1,k]}\oplus V^{[1,k]}$ intersect trivially because $f_1,\ldots,f_k$ are linearly independent.
  As above the lower horizontal sequence also splits, establishing the claim for $I=[1,k]$.
\end{proof}


%%%%%%%%%%%%%%%%%%%%%%%%%%%%%%%%%%%%%%%%%%%%%%%%%%%%
\section{Quiver Grassmannians of $\widetilde{K(n)}$}

Let $\tilde P_m$ be a fixed lift of $P_m$ to the universal cover $\widetilde{K(n)}$.
\begin{lemma}
  \label{le:preprojective lifts}
  There exist lifts $\tilde P_{m-1,i}$ for $1\le i\le n$ of $P_{m-1}$ to $\widetilde{K(n)}$ so that:
  \begin{enumerate}
    \item $\Hom_Q(P_{m-1},P_m)\cong\bigoplus_{i=1}^n \Hom_{\tilde Q}(\tilde P_{m-1,i},\tilde P_m)$, where each $\Hom_{\tilde Q}(\tilde P_{m-1,i},\tilde P_m)$ is one-dimensional;
    \item For any proper subset $\{i_1,\ldots,i_k\}\subset\{1,\ldots,n\}$, there exists a short exact sequence
      \[0\longrightarrow \tilde P_{m-1,i_1}\oplus\cdots\oplus\tilde P_{m-1,i_k}\longrightarrow\tilde P_m\longrightarrow \tilde P_m^{i_1,\ldots,i_k}\longrightarrow 0;\]
    \item The lifts $\tilde P_{m-1,i}$ are pairwise orthogonal. 
    \item All nontrivial proper subrepresentations of $\tilde P_m^{i_1,\ldots,i_k}$ are preprojective.
  \end{enumerate}
\end{lemma}

We will always choose the subset $\{1,\ldots,k\}$ when using Lemma~\ref{le:preprojective lifts}.2 and thus we denote the cokernel simply by $\tilde P_m^{(k)}$.

\begin{lemma}
  If each $Gr^{\tilde Q}_\tbfe(\tilde P_{m-1,i})$ has a cell decomposition, then $Gr^{\tilde Q}_\tbfe(\bigoplus\tilde P_{m-1,i_j})$ has a cell decomposition.
\end{lemma}

\begin{lemma}
  \begin{enumerate}
    \item $\tilde P_m^{(n-1)}\cong\tilde P_{m-1}^{(1)}$
    \item The subrepresentation $\bigoplus_{i=1}^{k-1}\tilde P_{m-1,i}\oplus\bigoplus_{i=1}^k\tilde P_{m-2,i}\subset\bigoplus_{i=1}^k\tilde P_{m-1,i}$ is in $(\tilde P_m^{(k)})^\perp$ and
      \[\Ext(\bigoplus_{i=1}^k\tilde P_{m-1,i},\tilde P_m^{(k)})\cong\Ext(\tilde P_{m-1}^{(k)},\tilde P_m^{(k)})\]
      where $\tilde P_{m-1}^{(k)}$ above denotes the cokernel of the inclusion.
  \end{enumerate}
\end{lemma}
\begin{lemma}\label{AR}
Consider $\ses{\tilde P_m}{\tilde P_{m+1}^{(k)}}{\tilde P_{m+1}^{(k-1)}}$
There exists a short exact sequence 
\[\ses{\tilde P_{m-1}^{n-1}\oplus \tilde P_{m-2}^{n-k}}{\tilde P_m}{\tau\tilde P_{m+1}^{(k-1)}}\]
where $\tau$ denotes the Auslander-Reiten translate. Moreover, we have $\Ext(\tilde P_{m+1}^{(k-1)},\tilde P_m)\cong \Ext(\tilde P_{m+1}^{(k-1)},\tau\tilde P_{m+1}^{(k-1)})=k$ and $\Hom(\tilde P_{m+1}^{(k-1)},\tau\tilde P_{m+1}^{(k-1)})=0$.
\footnote{introduce/consider/check indices, we have to mod out the corresponding reps in the covering} 
\end{lemma}
\begin{proof}
Idea: Check this for $m=1$ or $m=2$ and apply BGP-reflections.
\end{proof}
\begin{lemma}\label{quotient}
 Consider $\ses{\tilde P_m}{\tilde P_{m+1}^{(k)}}{\tilde P_{m+1}^{(k-1)}}$. If $U\subset \tilde P_m$ such that $\Ext(\tilde P_{m+1}^{(k-1)},U)\neq 0$, then we have $\Ext(\tilde P_{m+1}^{(k-1)},\tilde P_m/U)= 0$.
\end{lemma}
\begin{proof}
Idea: Every $U\subset \tilde P_m$ gives rise to a diagram
  \[\xymatrix{0\ar[r] &\tilde P_{m-1}^{n-1}\oplus \tilde P_{m-2}^{n-k}\ar[r] &  \tilde P_m\ar[r] & \tau\tilde P_{m+1}^{(k-1)}\ar[r] & 0 \\
  0\ar[r]& V\ar[r]\ar[u] & U\ar[u]\ar[r]& W\ar[r]\ar[u]& 0}\]
	
	If $\Ext(\tilde P_{m+1}^{(k-1)},U)\neq 0$, we have $W\neq 0$. Thus $\tau\tilde P_{m+1}^{(k-1)}/W$ is a proper factor. It follows that we have $\Ext(\tilde P_{m+1}^{(k-1)},\tau\tilde P_{m+1}^{(k-1)}/W)=0$ by Auslander-Reiten-theory and because 
$\Ext(\tilde P_{m+1}^{(k-1)},\tau\tilde P_{m+1}^{(k-1)})=k$. 

Since we have $\Ext(\tilde P_{m+1}^{(k-1)},\tilde P_{m-1}^{n-1}\oplus \tilde P_{m-2}^{n-k})=0$, it follows $\Ext(\tilde P_{m+1}^{(k-1)},(\tilde P_{m-1}^{n-1}\oplus \tilde P_{m-2}^{n-k})/V)=0$. The first statement should be a consequence of Lemma \ref{AR}. This yields the claim.

%First observe that $\tilde P_m\in ~^\perp \tilde P_{m+1}^{(k-1)}$ and $\Ext(\tilde P_{m+1}^{(k-1)},\tilde P_m)=k$. Assume that $U$ is preprojective with $\Hom(U,\tilde P_{m+1}^{(k-1)})=0$ (check again). This yields $\Ext(\tilde P_m/U,\tilde P_{m+1}^{(k-1)})=0$. Consider
%\[0\longrightarrow[\tilde P_{m+1}^{(k-1)},\tilde P_m/U]\longrightarrow[\tilde P_{m+1}^{(k-1)},U]^1\xlongrightarrow{\pi}[\tilde P_{m+1}^{(k-1)},\tilde P_m]^1\longrightarrow[\tilde P_{m+1}^{(k-1)},\tilde P_m/U]^1\longrightarrow 0.\]
%Since $\Ext(\tilde P_{m+1}^{(k-1)},\tilde P_m)=k$, we have that $\pi$ is surjective if $\pi\neq 0$. If $U=U_1\oplus\ldots\oplus U_r$ is a decomposition into indecomposable representations, we can write $\pi=(\pi_1,\ldots,\pi_r)$ and it suffices to show that $\pi_j\neq 0$ for at least one $j=1,\ldots,r$. Thus we can without loss of generality assume that $U$ is indecomposable (check again). This means that $U\cong \tilde P_{i}$ for some $i=1,\ldots,m$. Moreover, $\tilde P_{m+1}/U$ is indecomposable (details).
%
%If $\tilde P_{m+1}/U$ is exceptional, $\Ext(\tilde P_m/U,\tilde P_{m+1}^{(k-1)})=0$ together with \cite[Theorem 4.1]{sch} yields that either $\Ext(\tilde P_{m+1}^{(k-1)},\tilde P_m/U)=0$ or $\Hom(\tilde P_{m+1}^{(k-1)},\tilde P_m/U)=0$. Thus $\Ext(\tilde P_{m+1}^{(k-1)},\tilde P_m/U)\neq 0$ would imply that $\Ext(\tilde P_{m+1}^{(k-1)},U)=0$. But this contradicts the assumption and the claim follows.
%
%Thus assume that $\tilde P_{m+1}/U$ has self-extensions. Since $\Ext(\tilde P_m/U,\tilde P_{m+1}^{(k-1)})= 0$, from \cite[Lemma 4.1]{hr} we obtain that every homomorphism $\tilde P_{m+1}^{(k-1)}\to \tilde P_m/U$ is injective or surjective. By a dimension count it needs to be surjective (case $k=1$?). Such a homomorphism would yield $\Ext(\tilde P_m/U,\tilde P_{m+1}^{(k-1)})\twoheadrightarrow\Ext(\tilde P_m/U,\tilde P_m/U)\neq 0$ which contradicts the further observations. Thus $\Ext(\tilde P_{m+1}^{(k-1)},U)\neq 0$, already yields $\Ext(\tilde P_{m+1}^{(k-1)},\tilde P_m/U)= 0$ in this case.
%
%Case $\Hom(U,P)\neq 0$?

\end{proof}

\begin{proposition}
  Consider $\psi:Gr^{\tilde Q}_\tbfe(\tilde P_m)\to\bigsqcup_{\tbff+\tbfg=\tbfe} Gr^{\tilde Q}_\tbff(\bigoplus_{i=1}^k \tilde P_{m-1,i})\times Gr^{\tilde Q}_\tbfg(\tilde P_m^{(k)}$.  Then the following hold:
  \begin{enumerate}
    \item For $V\subsetneq \tilde P_m^{(k)}$ and $U\subset\bigoplus_{i=1}^k\tilde P_{m-1,i}$, we have $\psi^{-1}(U,V)=\AA^{\langle V,\bigoplus_{i=1}^k\tilde P_{m-1,i}/U\rangle}$.
    \item If $V=\tilde P_m^{(k)}$, the fibre $\psi^{-1}(U,V)$ is non-empty and of constant dimension if and only if $\Ext(\tilde P_m^{(k)},U)\neq 0$.
  \end{enumerate}
\end{proposition}
\begin{proof}
  (1) $V$ is preprojective but $\bigoplus_{i=1}^k\tilde P_{m-1,i}/U$ is not unless $U=0$

  (2) 
If $\Ext(\tilde P_m^{(k)},U)= 0$, the fibre is empty because the sequence is non-split. If $\Ext(\tilde P_m^{(k)},U)\neq 0$, Lemma \ref{quotient} yields $\Ext(\tilde P_m^{(k)},\tilde P_{m-1}/U)=0$. In particular, the dimension is constant if it is not empty and
$\Ext(\tilde P_{m+1}^{(k-1)},U)\xlongrightarrow{\pi}\Ext(\tilde P_{m+1}^{(k-1)},\tilde P_m)$ is surjective. But this already means that the fibre is not empty.
\end{proof}

\begin{theorem}
  Every quiver Grassmannian of a preprojective or preinjective representation of $K(n)$ and $\widetilde{K(n)}$ has a cell decomposition.
\end{theorem}

\begin{question}
  cells of $Gr^{\tilde Q}_\tbfe(\tilde P_m)$ are in one-to-one correspondence with certain tuples of subgraphs for smaller $\tilde P_\ell^{i_1,\ldots,i_k}$
\end{question}


%%%%%%%%%%%%%%%%%%%%%%%%%%%%%%%%%%%%%%%%%%%%%%%%%%%%%%%%%%
\section{Compatible Pairs Label Cells in $Gr^Q_\bfe(P_m)$}
\begin{thebibliography}{10}
\bibitem{ass}
Assem, I., Simson, D., Skowronski, A.: Elements of the Representation Theory of Associative Algebras. Cambridge University Press, Cambridge 2007.
\bibitem{bgp}
Bernstein, I.~N., Gelfand, I.~M., Ponomarev, V.~A.: Coxeter functors, and Gabriel's theorem. Russian Mathematical Surveys \textbf{28}(2), 17-32 (1973).
\bibitem{hr} Happel, D., Ringel, C.M.: Tilted Algebras. Transactions of the American Mathematical Society {\bf 274}, no.2, 399-443 (1982).

\bibitem{sch} Schofield, A.: General representations of quivers. Proceedings of the London Mathematical Society (3) \textbf{65}(1), 46-64 (1992).

\end{thebibliography}

\end{document}
