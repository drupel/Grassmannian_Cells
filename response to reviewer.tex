\documentclass[titlepage,11pt,a4paper]{{scrartcl}}
\usepackage{amstext}
\usepackage{amssymb}
\usepackage{amsfonts}
\usepackage{amssymb}
\usepackage{amsmath}
\usepackage{graphicx}
\usepackage[arrow,matrix,curve]{xy}
\usepackage[latin1]{inputenc}
\usepackage{theorem}
\usepackage{geometry}
\usepackage{enumitem}
\geometry{a4paper,top=3cm, bottom=3cm} 
\newtheorem{satz}{Theorem}
\newtheorem{aufg}[satz]{Aufgabe}
\newtheorem{defi}[satz]{Definition}
\newtheorem{bem}[satz]{Remark}
\newtheorem{kor}[satz]{Corollary}
\newtheorem{lem}[satz]{Lemma}
\newtheorem{ver}[satz]{Conjecture}
\newtheorem{zit}[satz]{Zitat}
\newtheorem{beo}[satz]{Beobachtung}
\newtheorem{bei}[satz]{Example}
\newtheorem{pro}[satz]{Proposition}
\newcommand{\dju}{\bigcup\hspace{-2.23mm\dot}\hspace{2mm}}
\newcommand{\modu}{~\rm{mod}~}
\newcommand{\smdju}{\cup\hspace{-2.15mm\dot}\hspace{2mm}}
\newcommand\dis[1]{\bigcup_{#1}\hspace{-2.5mm\dot}\hspace{2mm}}
\newcommand{\qed}{\begin{flushright}$\square$\end{flushright}}
\newcommand\R{$\mathbb{R}$}
\newcommand\Q{$\mathbb{Q}$}
\newcommand\Z{$\mathbb{Z}$}
\newcommand\N{$\mathbb{N}$}
\newcommand\C{$\mathbb{C}$}
\newcommand\Rn{\mathbb{R}}
\newcommand\Qn{\mathbb{Q}}
\newcommand\Zn{\mathbb{Z}}
\newcommand\Nn{\mathbb{N}}
\newcommand\Cn{\mathbb{C}}
\newcommand\Sk{\langle\,\,,\,\rangle}
\newcommand{\Hom}{\mathrm{Hom}} 

\newcommand{\filt}[6]{
\[
\begin{xy}
\xymatrix@R20pt@C20pt{
&\mathbb{C}^3&\\\langle #1,#2 \rangle\ar[ru]&\langle #3,#4\rangle\ar[u]&\langle #5,#6\rangle\ar[lu]\\\langle #1\rangle\ar[u]&\langle #3\rangle\ar[u]&\langle #5\rangle\ar[u]}
\end{xy}
\]
} 
\begin{document}
\pagestyle{empty}


\noindent{\bf Response to the reviewer - Cell decompositions for rank two quiver Grassmannians}\\



\noindent In the following, we point out the changes we made.

Section 4:
\begin{enumerate}
\renewcommand{\labelenumi}{(\arabic{enumi})}
\item Note that the proof and the examples before show how to compute the kernel of the right hand vertical map for $m=2$ and $m=3$. As it is indeed not obvious to prove surjectivity for general $m$ directly, we deduce it from the $m=2,3$-cases and apply $\tau^{-1}$-shifts which preserve exact sequences if no direct summand is injective. This is mentioned in Remark 3.2 to which we refer to in the proof of Lemma 3.30.
\item We omit the notation $~\perp$ as it is only used in Corollary 3.32.
\item This should be a matter of definition and by reindexing appropriately we should get the version with $\ast$'s on the right.
\item We refer to Cerulli Irelli's paper now. But not that the assumptions in Theorem 1 are stronger than ours as we do not assume that every basis element is mapped on a scalar multiple of another basis element. Note also that the number of fixed points is finite in Cerulli Irelli's case which is not true in general.
\item We inserted a remark concerning the result of Haupt. Our approach seems to simplify the language and generalize the techniques of Haupt. If we are not mistaken, the proof of Theorem 1.2(c) in Section 8 only deals with those representations of the universal (abelian) covering whose dimension vectors only consists of $0$'s and $1$'s. We actually consider arbitrary representations of $\hat Q$. Moreover, the explicit description of the fixed points of $\mathrm{Gr}_{\mathbf{e}}(M)$ of a liftable representation $M$ as subrepresentations of the lifted representations $\hat M$ seems to be missing.  

\item We inserted a remark which addresses this issue.
\item We inserted a remark in Theorem 4.22 (former 4.20) which is meant to explain how the statement is obtained from the claim. Moreover, it should make the proof clearer.
\item I find the strategy if 4.23 is very clear and I also think that the proof of 4.24 is very close to the suggestion of the referee. Any ideas what to change?
\end{enumerate}


\end{document}
